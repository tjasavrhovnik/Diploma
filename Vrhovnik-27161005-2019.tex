\documentclass[mat1]{fmfdelo}
% \documentclass[fin1]{fmfdelo}
% \documentclass[isrm1]{fmfdelo}
% \documentclass[mat2]{fmfdelo}
% \documentclass[fin2]{fmfdelo}
% \documentclass[isrm2]{fmfdelo}

% naslednje ukaze ustrezno napolnite
\avtor{Tjaša Vrhovnik}

\naslov{Fareyevo zaporedje in Riemannova hipoteza}
\title{The Farey Sequence and The Riemann Hypothesis}

% navedite ime mentorja s polnim nazivom: doc.~dr.~Ime Priimek,
% izr.~prof.~dr.~Ime Priimek, prof.~dr.~Ime Priimek
% uporabite le tisti ukaz/ukaze, ki je/so za vas ustrezni
\mentor{izr.~prof.~dr.~Aleš Vavpetič}
% \mentorica{}
% \somentor{}
% \somentorica{}
% \mentorja{}{}
% \mentorici{}{}

\letnica{2019} % leto diplome

%  V povzetku na kratko opišite vsebinske rezultate dela. Sem ne sodi razlaga organizacije dela --
%  v katerem poglavju/razdelku je kaj, pač pa le opis vsebine.
\povzetek{}

%  Prevod slovenskega povzetka v angleščino.
\abstract{}

% navedite vsaj eno klasifikacijsko oznako --
% dostopne so na www.ams.org/mathscinet/msc/msc2010.html
\klasifikacija{}
\kljucnebesede{} % navedite nekaj ključnih pojmov, ki nastopajo v delu
\keywords{} % angleški prevod ključnih besed

\zapisiMetaPodatke  % poskrbi za metapodatke in veljaven PDF/A-1b standard

% aktivirajte pakete, ki jih potrebujete
% \usepackage{tikz}

% za številske množice uporabite naslednje simbole
\newcommand{\R}{\mathbb R}
\newcommand{\N}{\mathbb N}
\newcommand{\Z}{\mathbb Z}
\newcommand{\C}{\mathbb C}
\newcommand{\Q}{\mathbb Q}

% matematične operatorje deklarirajte kot take, da jih bo Latex pravilno stavil
% \DeclareMathOperator{\conv}{conv}

% vstavite svoje definicije ...
%  \newcommand{}{}

\begin{document}

%%%%%%%%%%%%%%%%%%%%%%%%%%%%%%%%%%%%%%%%%%%%%%%%%%%%%%%%%%%%%%%%%%%%%%%%%%%%%%%%%%%%%%%%%%%%%%%%%%%%%%%%%%%%%%%%%%%%%%%%%%%%%%%%%%%%%%%%%%%%%%%%%%
%
\section{Uvod}

Zgodovina Fareyevega zaporedja sega v London 18. stoletja. Med letoma 1704 in 1841 je izhajal letni zbornik \emph{The Ladies Diary: or, the Woman's Almanack}, ki je povezoval ljubitelje matematičnih ugank. Bralce so namreč nagovarjali k pošiljanju in reševanju aritmetičnih problemov, ki so bili v zborniku objavljeni. Leta 1747 se je pojavilo naslednje vprašanje: Najti je potrebno število ulomkov različnih vrednosti, manjših od $1$, katerih imenovalec ni večji od $100$.

%%%%%%%%%%%%%%%%%%%%%%%%%%%%%%%%%%%%%%%%%%%%%%%%%%%%%%%%%%%%%%%%%%%%%%%%%%%%%%%%%%%%%%%%%%%%%%%%%%%%%%%%%%%%%%%%%%%%%%%%%%%%%%%%%%%%%%%%%%%%%%%%%%
%
\section{Fareyevo zaporedje}

%%%%%%%%%%%%%%%%%%%%%%%%%%%%%%%%%%%%%%%%%%%%%%%%%%%%%%%%%%%%%%%%%%%%%%%%%
%
\subsection{Zgodovina Fareyevega zaporedja}

Vrnimo se k v uvodu omenjeni nalogi o številu ulomkov različnih vrednosti, manjših od $1$, z imenovalci kvečjemu $100$. Prvi odgovor na članek je bila tabela ulomkov z imenovalci manjšimi od $10$, nato pa še dve tabeli z rezultatoma $3055$ in $4851$. Leta 1751 je R.~Flitcon objavil pravilni odgovor $3003$, kateremu je dodal tudi opis postopka. Preden ga razložimo, si oglejmo Eulerjevo funkcijo in njene lastnosti, ki nas bo pripeljala do rešitve. 
%
% Eulerjeva funkcija

\begin{definicija}
Preslikava \( \varphi \colon \mathbb{N} \rightarrow \mathbb{N}\), ki za vsako naravno število $n$ prešteje števila, manjša od $n$, ki so $n$ tuja, se imenuje \emph{Eulerjeva funkcija} $\varphi$.
\end{definicija}

\begin{trditev}
\label{trd:MultipEuler}
Če sta $k$ in $l$ tuji si števili, velja $\varphi (kl) = \varphi (k) \varphi (l)$, torej je Eulerjeva funkcija multiplikativna.
\end{trditev}

\begin{dokaz}
V dokazu multiplikativnosti si bomo pomagali z lastnostmi grup. 
Naj $\mathbb{Z}_{k}^\ast $ označuje grupo vseh obrnljivih elementov grupe $\mathbb{Z}_{k}$. Vemo, da so obrnljivi elementi grupe $\mathbb{Z}_{k}$ tista števila iz množice \( \{0, 1, \ldots, k-1 \}, \) ki so tuja $k$, zato je $|\mathbb{Z}_{k}^\ast| = \varphi(k).$  
Dobimo zvezi
\[ |\mathbb{Z}_{kl}^\ast| = \varphi(kl), \]
\[ |\mathbb{Z}_{k}^\ast| |\mathbb{Z}_{l}^\ast| = \varphi(k) \varphi(l).\]
Znano je, da je preslikava \( \psi \colon \mathbb{Z}_{kl}^\ast \rightarrow \mathbb{Z}_{k}^\ast \times \mathbb{Z}_{l}^\ast \) za tuji naravni števili $k$ in $l$ izomorfizem grup. Ker je moč kartezičnega produkta dveh množic enaka produktu njunih moči, sledi 
\[ \varphi(kl) = |\mathbb{Z}_{kl}^\ast| = |\mathbb{Z}_{k}^\ast| |\mathbb{Z}_{l}^\ast| = \varphi(k) \varphi(l), \]
s čimer je multiplikativnost dokazana.
\end{dokaz}

\begin{trditev}
\label{trd:EulerPrastProd}
Vrednost Eulerjeve funkcije je enaka
\[ \varphi(n) = n \prod_{p|n} \left (1 - \frac{1}{p} \right ), \] kjer so $p$ prafaktorji števila $n$.
\end{trditev}

\begin{dokaz}
Zapišimo $n$ kot produkt prafaktorjev, \(n=\prod_{i=1}^m p_i^{r_i}\), kjer so $r_i \in \mathbb{N}$ in $p_i$ praštevila.
Funkcija $\varphi(p^{r})$ prešteje vsa števila, manjša od $p^r$, ki so tuja $p^r$. To so natanko tista, ki niso deljiva s praštevilom $p$. Večkratnikov $p$ med števili $1, 2, \ldots, p^r-1$ je toliko kot večkratnikov $p$ med števili $1, 2, \ldots, p^r$, teh pa je \( \frac{p^r}{p} = p^{r-1}. \)
Torej je 
\[ \varphi(p^r) = (p^r - 1) - (p^{r-1} - 1) = p^r - p^{r-1} = p^r \left (1 - \frac{1}{p} \right). \] 
Z upoštevanjem multiplikativnosti funkcije $\varphi$ dobimo
%
\begin{align*}
 \varphi(n) 
 &= \varphi \left (\prod_{i=1}^m p_i^{r_i} \right ) = \prod_{i=1}^m \varphi (p_i^{r_i} ) = 
 \prod_{i=1}^m p_i^{r_i} \left (1 - \frac{1}{p_i} \right ) \\
 &= \prod_{i=1}^m p_i^{r_i} \times \prod_{i=1}^m \left (1 - \frac{1}{p_i} \right ) = 
 n \prod_{i=1}^m \left (1 - \frac{1}{p_i} \right ) = n \prod_{p|n} \left (1 - \frac{1}{p} \right ),
\end{align*}
kar smo želeli dokazati.
\end{dokaz}
%
% Flitconova rešitev

\begin{trditev}
Obstajajo $3003$ racionalna števila $\frac{p}{q}$, za katera velja $0<\frac{p}{q}<1$ ter je $q \leq 100$.
\end{trditev}

Namesto formalnega dokaza trditve bomo predstavili Flitconovo rešitev. Naredimo tabelo s tremi stolpci in 99 vrsticami. V prvi stolpec vsake vrstice napišemo po eno izmed naravnih števil od $2$ do $100$. V drugi stolpec posamezne vrstice zapišemo naravno število iz prvega stolpca kot produkt prafaktorjev, v tretji stolpec pa vrednost $\varphi(n)$. Pomagamo si s trditvama~\ref{trd:MultipEuler} in \ref{trd:EulerPrastProd}. Vsota vrednosti v tretjem stolpcu nam da število iskanih ulomkov. Res, vsak $\varphi(n)$ nam pove število okrajšanih ulomkov med $0$ in $1$ z imenovalcem $n$, vsota vrednosti Eulerjeve funkcije $\varphi(n)$ za vsa števila $n$ med $2$ in $100$ pa število vseh okrajšanih ulomkov med $0$ in $1$ z imenovalci med $2$ in $100$. \footnote{Čeprav Flitcon ne omenja Eulerjeve funkcije $\varphi$, je uporabil njene lastnosti v svoji matematično manj formalni metodi.}

Neodvisno od Flitconove rešitve je francoski matematik Charles Haros leta 1802 sestavil enak seznam ulomkov, vendar na precej bolj zanesljiv način. Haros se dela ni lotil z željo po reševanju aritmetične naloge, pač pa je pisal tabele za pretvarjanje med ulomki in decimalnim zapisom ter obratno. V Franciji so namreč v času revolucije konec 18. stoletja uvajali nov metrični sistem, ki je med drugim zahteval uporabo decimalnega zapisa. Tabele so bile objavljene v časniku \emph{Journal de l'Ecole Polytechnique}, primerom ter algoritmom za pretvarjanje pa so bile dodane skice dokazov in nekatere lastnosti zaporedja ulomkov, ki so kasneje postali znani pod imenom Fareyevo zaporedje. 

Posebej zanimiva je zgodba o pivovarju in ljubiteljskemu matematiku Henryju Goodwynu. Čeprav ni imel formalne izobrazbe, se je navduševal nad znanostjo in tehniko, sestavljal različne tabele in računal, kako izboljšati svoje poslovanje. Po upokojitvi se je vse bolj posvečal matematiki -- tako je med letoma 1816 in 1823 objavil več člankov s tabelami okrajšanih ulomkov. Njegovo delo sta opazila znameniti francoski matematik Augustin Louis Cauchy in John Farey, geolog, po komer se obravnavano zaporedje okrajšanih ulomkov imenuje. Vemo, da je Cauchy prispeval nekaj dokazov lastnosti Fareyevega zaporedja, v nasprotju pa ostaja neznano, ali sta Goodwyn in Farey zaporedje in nekatere njegove lastnosti odkrila neodvisno od Harosa, bodisi sta vedela za njegove ugotovitve. Farey je najverjetneje na podlagi Goodwynovih tabel maja 1816 v pismu časopisu \emph{The Philosophical Magazine and Journal} z naslovom \emph{On a curious Property of vulgar Fractions} predstavil medianto, najpomembnejšo lastnost zaporedja. Čeprav zaporedje morda neupravičeno nosi ime Johna Fareya, pa ne smemo spregledati njegovega prispevka k raziskovanju matematike v glasbi, vzorcev, astronomije in seveda geologije.

%%%%%%%%%%%%%%%%%%%%%%%%%%%%%%%%%%%%%%%%%%%%%%%%%%%%%%%%%%%%%%%%%%%%%%%%%
% Fareyevo zaporedje
%
\subsection{O Fareyevem zaporedju}

\begin{definicija}
\label{def:Farey}
\emph{Fareyevo zaporedje reda n} oz.\ \emph{n-to Fareyevo zaporedje} je množica racionalnih števil $\frac{p}{q}$ urejenih po velikosti, kjer sta $p$ in $q$ tuji si števili, ter velja $0 \leq p \leq q \leq n$. Označimo ga z $F_n$.

Ekvivalentno, $F_n$ vsebuje vse okrajšane ulomke med $0$ in $1$ z imenovalci, kvečjemu enakimi $n$.
\end{definicija}

\begin{primer}
Poglejmo si prvih nekaj Fareyevih zapredij.

\(F_1 = \left \{\frac{0}{1}, \frac{1}{1} \right \} \)

\(F_2 = \left \{\frac{0}{1}, \frac{1}{2}, \frac{1}{1} \right \} \)

\(F_3 = \left \{\frac{0}{1}, \frac{1}{3}, \frac{1}{2}, \frac{2}{3}, \frac{1}{1} \right \} \)

\(F_4 = \left \{\frac{0}{1}, \frac{1}{4}, \frac{1}{3}, \frac{1}{2}, \frac{2}{3}, \frac{3}{4}, \frac{1}{1} \right \} \)

\(F_5 = \left \{\frac{0}{1}, \frac{1}{5}, \frac{1}{4}, \frac{1}{3}, \frac{2}{5}, \frac{1}{2}, \frac{3}{5}, \frac{2}{3}, \frac{3}{4}, \frac{4}{5}, \frac{1}{1}\right \} \)
\end{primer}

\begin{opomba}
Če pogoj $0 \leq p \leq q \leq n$ v definiciji~\ref{def:Farey} omilimo v pogoj $0 \leq p,q \leq n$, okrajšane ulomke z intervala $[0,1]$ razširimo na interval $[0, \infty)$. V primeru, ko za $p$ in $q$ dovoljujemo tudi negativna cela števila, dobimo okrajšane ulomke na celotni realni osi.
\end{opomba}

V zgornjih primerih opazimo, da za vsaka sosednja člena Fareyevega zaporedja velja naslednje. Če števec prvega ulomka množimo z imenovalcem drugega in nato vlogi ulomkov zamenjamo, je razlika obeh produktov po absolutni vrednosti enaka $1$. To se bo izkazalo za pomembno opazko, zato vpeljemo pojem, ki sledi.

\begin{definicija}
Sosednja člena v Fareyevem zaporedju imenujemo \emph{Fareyeva soseda}.
\end{definicija}

% medianta
%
\begin{definicija}
Naj bosta $\frac{a}{b}$ in $\frac{c}{d}$ sosednja člena nekega Fareyevega zaporedja. Člen \[\frac{a+c}{b+d} \] imenujemo \emph{medianta}.
\end{definicija}

\begin{trditev}
Za medianto okrajšanih ulomkov \(\frac{a}{b} < \frac{c}{d}\) velja  \(\frac{a}{b} < \frac{a+c}{b+d} < \frac{c}{d}\) .
\end{trditev}

\begin{dokaz}
Poračunajmo razliki med členoma
\[\frac{a+c}{b+d} - \frac{a}{b} = \frac{ab+bc-ab-ad}{b(b+d)} = \frac{bc-ad}{b(b+d)} > 0\] in
\[\frac{c}{d} - \frac{a+c}{b+d} = \frac{bc+cd-ad-cd}{d(b+d)} = \frac{bc-ad}{d(b+d)} > 0.\]
Obe neenakosti sledita iz dejstva, da je \(\frac{a}{b} < \frac{c}{d}\), kjer so \(a, b, c, d \in \mathbb{N} \), zato je \( ad < bc.\)
Zveza torej velja. 
\end{dokaz}

% lastnosti 
%
Kako dobimo člen Fareyevega zaporedja reda $(n+1)$?
Označimo iskani okrajšan ulomek s $\frac{k}{n+1}$. Seveda sta $k, n \in\mathbb{N}, k < n+1$ tuji si števili. Zato obstajata enolično določeni naravni števili $a < b$, da velja $a(n+1)-bk=1.$ S preoblikovanjem zadnje enakosti dobimo zvezo $a(n+1-b)-b(k-a)=1,$ kar pomeni, da sta si tudi naravni števili $k-a$ in $n+1-b$ tuji. Brez škode za splošnost naj bo $k-a<n+1-b.$ Zato lahko tvorimo okrajšan ulomek $\frac{k-a}{n+1-b}$, ki pripada nekemu Fareyevemu zaporedju. Prav tako je okrajšan ulomek $\frac{a}{b}$ element nekega Fareyevega zaporedja. Sedaj prepišimo ulomek $\frac{k}{n+1}$ v $\frac{a+(k-a)}{b+(n+1-b)},$ kar pa je medianta ulomkov $\frac{a}{b}$ in $\frac{k-a}{n+1-b}.$ Dokazali smo naslednjo lemo.

\begin{lema}
Dano naj bo Fareyevo zaporedje. Elemente zaporedja višjega reda dobimo z računanjem mediant elementov danega zaporedja.
\end{lema}

\begin{trditev}
Naj velja \( 0 \leq \frac{a}{b} < \frac{c}{d} \leq 1\). Ulomka $\frac{a}{b}$ in $\frac{c}{d}$ sta Fareyeva soseda v nekem Fareyevem zaporeju natanko tedaj, ko velja \(bc - ad = 1\).
\end{trditev}

\begin{dokaz}
$(\Rightarrow)$ Dokaz bo potekal z indukcijo na $n$.
Za $n=1$ je $F_n = \{\frac{0}{1}, \frac{1}{1}\}$, $bc - ad = 1\cdot1 - 0\cdot1 = 1$, zato osnovni korak velja.
Po indukcijski predpostavki za zaporedje \( F_n = \{\ldots \frac{a}{b}, \frac{c}{d} \ldots \} \) velja $bc - ad = 1$. Dokažimo, da velja tudi za $F_{n+1}$. Vemo, da nove člene zaporedja dobimo z računanjem mediant. Če je $b+d > n+1,$ potem $\frac{a+c}{b+d} \notin F_{n+1}$ in je \( F_{n+1} = \{\ldots \frac{a}{b}, \frac{c}{d} \ldots \} \) ter po indukcijski predpostavki $bc - ad = 1$. 
Če je $b+d < n+1,$ je $\frac{a+c}{b+d}$ že nek člen v zaporedju $F_n$ in uporabimo indukcijsko predpostavko. 
Če je $b+d = n+1$, je \( F_{n+1} = \{\ldots \frac{a}{b}, \frac{a+c}{b+d}, \frac{c}{d} \ldots \} \) in $b(a + c) - a(b + d) = ba + bc - ab - ad = bc - ad = 1$, kjer smo v zadnji enakosti uporabili indukcijsko predpostavko. Podobno je $(b + d)c - (a + c)d = bc + dc - ad - cd = bc - ad = 1$. Indukcijski korak je s tem končan. Torej sklep velja za vsa Fareyeva zaporedja.

$(\Leftarrow)$ Obratno, naj bodo $\frac{a}{b}$, $\frac{p}{q}$ in $\frac{c}{d}$ členi poljubnega Fareyevega zaporedja, za katere velja $\frac{a}{b} <\frac{p}{q} < \frac{c}{d}$ in $bp - aq = qc - pd = 1$. S preureditvijo te enakosti dobimo
\[ bp + pd = aq + qc, \]
\[ p(b + d) = q(a + c), \]
\[ \frac{p}{q} = \frac{a+c}{b+d}. \]
Vidimo, da je $\frac{p}{q}$ medianta ulomkov $\frac{a}{b}$ in $\frac{c}{d},$ od tod pa sledi, da sta $\frac{a}{b}$ in $\frac{p}{q}$ ter $\frac{p}{q}$ in $\frac{c}{d}$ Fareyeva soseda.
\end{dokaz}

\begin{lema}
\label{lema:MediantaOkrUlom}
Medianta je okrajšan ulomek.
\end{lema}

\begin{dokaz}
Naj za $\frac{a}{b} < \frac{c}{d}$ velja $bc - ad = 1$. Dokazati želimo, da je $\frac{a+c}{b+d}$ okrajšan ulomek, z drugimi besedami, da sta si števili $a+c$ in $b+d$ tuji. Če preoblikujemo zgornjo enakost, dobimo 
\[ 1 = bc - ad = ba + bc - ab - ad = b(a + c) - a(b + d), \]
kar pomeni, da $b+d$ in $a+c$ nimata skupnega faktorja. Ulomek $\frac{a+c}{b+d}$ je torej okrajšan.
\end{dokaz}

%%%%%%%%%%%%%%%%%%%%%%%%%%%%%%%%%%%%%%%%%%%%%%%%%%%%%%%%%%%%%%%%%%%%%%%%%
% dolžina zaporedja
%
\subsection{Dolžina Fareyevega zaporedja}

Flitconova metoda za izračun števila okrajšanih ulomkov nas pripelje do naslednje rekurzivne formule dolžine Fareyevega zaporedja.

\begin{trditev}
\label{trd:DolzinaZap}
Naj bo $\varphi$ Eulerjeva funkcija. Dolžina Fareyevega zaporedja reda n je
\[  |F_{n}| = |F_{n-1}| + \varphi(n). \]
\end{trditev}

\begin{opomba}
\label{op:AsimptotDolzina}
Z upoštevanjem vrednosti $|F_{1}| = 2$ iz trditve~\ref{trd:DolzinaZap} sledi \[  |F_{n}| = \sum_{i=1}^n \varphi(i) + 1. \]
\end{opomba}

\begin{trditev}
\label{trd:AsimptotDolzina}
Asimptotično se dolžina Fareyevega zaporedja obnaša kot
\[  |F_{n}|\sim\frac{3n^2}{\pi^2}. \]
\end{trditev}

\begin{opomba}
Simbol $\sim$ v trditvi~\ref{trd:AsimptotDolzina} označuje asimptotično ekvivalentno obnašanje dveh funkcij.
Po definiciji za funkciji $f(x)$ in $g(x)$ velja $f(x) \sim g(x)$ natanko tedaj, ko je $ \lim_{x \to \infty} \frac{f(x)}{g(x)} = 1$.
\end{opomba}

% definicije pred dokazom
%
Preden dokažemo zgornjo trditev, definirajmo naslednjo oznako in dve funkciji, ki jih bomo v dokazu potrebovali.

\begin{definicija}
\emph{Notacija veliki O} predstavlja množico funkcij, ki so po absolutni vrednosti do multiplikativne konstante manjše od dane funkcije.

Natančneje, funkcija $f$ pripada razredu $O(g)$, če obstajata taki konstanti $M$ in $x_{0}$, da za vsak $x > x_{0}$ velja $|f(x)| \leq M \cdot |g(x)|$.

Pišemo $f \in O(g)$ oziroma $f = O(g)$.
\end{definicija}

\begin{definicija}
Preslikava \( \mu\colon \mathbb{N} \to \mathbb{N} \), definirana kot
\[
\mu(n) = \left\{
\begin{array}{rl}
0 & ;\ \mbox{$n$ je deljiv s kvadratom praštevila}\\
(-1)^p & ;\  \mbox{$n$ je produkt $p$ različnih praštevil,}
\end{array}
\right.
\]
se imenuje \emph{M\"obiusova funkcija}.
\end{definicija}

\begin{primer}
Izračunajmo vrednosti M\"obiusove funkcije za nekaj naravnih števil.

\( \mu(1)=1 \)

\( \mu(2)=(-1)^{1}=-1=\mu(3) \)

\( \mu(4)=\mu(2^{2})=0 \)

\( \mu(6)=\mu(2\cdot3)=(-1)^{2}=1 \)

\( \mu(8)=\mu(2\cdot2^{2})=0 \)

\( \mu(18)=\mu(2\cdot3^{2})=0 \)
\end{primer}

\begin{definicija}
\emph{Riemannova funkcija zeta} je za
 $s\in\mathbb{C}\backslash\{1\}$
definirana kot
\[ \zeta(s) = \sum_{n=1}^{\infty}\frac{1}{n^s}. \]
\end{definicija}

% dokaz asimptotskega obnašanja
%
Sedaj lahko dokažemo trditev~\ref{trd:AsimptotDolzina}.

\begin{dokaz}
Asimptotično obnašanje bomo izračunali s pomočjo ocene vrednosti vsote \( \sum_{i=1}^n \varphi(i) \).
Spomnimo se, da je \[ \varphi(n) = n \prod_{p|n} \left (1 - \frac{1}{p} \right ) = n - \sum \frac{n}{p} + \sum \frac{n}{pp'} - \cdots , \]
kjer so $p$, $p'$ praštevilski delitelji števila $n$.  Z upoštevanjem M\"obiusove funkcije je zadnja vsota enaka
\[ \varphi(n) = n \sum_{d|n} \frac{\mu(d)}{d} .\]
Sedaj računajmo vsoto 
%
\begin{align*}
\sum_{i=1}^n \varphi(i)
  &= \sum_{i=1}^n i \sum_{d|i} \frac{\mu(d)}{d} = \sum_{dd'\leq n}d' \mu(d) = 
    \sum_{d=1}^n \mu(d) \sum_{d'=1}^{\left \lfloor \frac{n}{d} \right \rfloor} d' \\
  &= \frac{1}{2} \sum_{d=1}^{n} \mu(d) \left (\left \lfloor \frac{n}{d} \right \rfloor ^2 + \left \lfloor \frac{n}{d} \right \rfloor \right) =
    \frac{1}{2} \sum_{d=1}^{n} \mu(d) \left (\frac{n^2}{d^2} + O \left (\frac{n}{d} \right) \right) \\
  &= \frac{1}{2}n^2 \sum_{d=1}^{n} \frac{\mu(d)}{d^2} + O \left (n \sum_{d=1}^{n} \frac{1}{d} \right ) \\
  &\stackrel{(1)}{=} \frac{1}{2}n^2 \sum_{d=1}^{\infty} \frac{\mu(d)}{d^2} - \frac{1}{2}n^2 \sum_{d=n+1}^{\infty} \frac{\mu(d)}{d^2} + O(n \ln{n}) \\
  &= \frac{1}{2}n^2 \sum_{d=1}^{\infty} \frac{\mu(d)}{d^2} + O \left (n^2\sum_{d=n+1}^{\infty} \frac{1}{d^2} \right ) + O(n \ln{n}) \\
  & \stackrel{(2)}{=} \frac{n^2}{2 \zeta(2)} + O(n) + O(n \ln{n}) \stackrel{(3)}{=} \frac{3n^2}{\pi^2} + O(n \ln{n}).
\end{align*}

V enakosti (1) smo zadnji sumand ocenili navzgor s pomočjo Taylorjevega razvoja funkcije $\ln$ kot
\[ \ln{(1+x)} = \sum_{n=1}^{\infty} (-1)^{n+1} \frac{x^n}{n}.\]

V enakosti (2) smo uporabili naslednjo oceno:
\[ n^2 \sum_{d=n+1}^{\infty} \frac{1}{d^2} \leq n^2 \sum_{d=n+1}^{\infty} \frac{1}{d(d-1)} = 
n^2 \sum_{d=n+1}^{\infty} \left (- \frac{1}{d} + \frac{1}{d-1} \right ) = n^2 \frac{1}{n} = n.\]

V enakosti (3) smo za izračun funkcije $\zeta(2)$ uporabili znano vrednost
\[ \zeta(2) = \sum_{n=1}^{\infty} \frac{1}{n^2} = \frac{\pi^2}{6}.\]

Po opombi~\ref{op:AsimptotDolzina} sledi, da je \(|F_n| = \frac{3n^2}{\pi^2} + O(n \ln{n}) \sim\frac{3n^2}{\pi^2}. \)
 %
\end{dokaz}

%%%%%%%%%%%%%%%%%%%%%%%%%%%%%%%%%%%%%%%%%%%%%%%%%%%%%%%%%%%%%%%%%%%%%%%%%%%%%%%%%%%%%%%%%%%%%%%%%%%%%%%%%%%%%%%%%%%%%%%%%%%%%%%%%%%%%%%%%%%%%%%%%%
%
\section{Fordovi krogi}

\begin{definicija}
\emph{Fordov krog} C($\frac{p}{q}$) je krog v zgornji polravnini, ki se abscisne osi dotika v točki $\frac{p}{q}$ in ima polmer $\frac{1}{2q^2}$. Pri tem sta $p$ in $q$ tuji si števili. 
\end{definicija}

Ker so Fordovi krogi definirani za vsak okrajšan ulomek, lahko poljubnemu racionalnemu številu enolično priredimo Fordov krog. Iz analize vemo, da je množica racionalnih števil gosta podmnožica množice realnih števil, abscisna os pa je geometrijska predstavitev le-te. Zato poljubno majhen interval na abscisni osi vsebuje neskončno mnogo dotikališč Fordovih krogov.

Zaradi simetrije je Fordove kroge dovolj obravnavati na intervalu $[0,1]$, obenem pa se zavedati, da jih lahko induktivno razširimo na celotno realno os.

\begin{trditev}
\label{trd:FordDisjTang}
Fordova kroga, ki pripadata različnima okrajšanima ulomkoma, sta bodisi tangentna bodisi disjunktna.
\end{trditev}

\begin{dokaz}
Naj za Fordova kroga C($\frac{a}{b}$) in C($\frac{c}{d}$) velja $\frac{a}{b} < \frac{c}{d}$.
Naj bodo A središče kroga C($\frac{a}{b}$), B središče kroga C($\frac{c}{d}$) in C presečišče navpične premice skozi točko B z vodoravno premico skozi točko A. Naj bosta še D in E presečišči daljice AB s Fordovima krogoma C($\frac{a}{b}$) in C($\frac{c}{d}$).

Vemo, da se kroga dotikata abscisne osi zaporedoma v točkah $\frac{a}{b}$ in $\frac{c}{d}$, njuna polmera pa merita $\frac{1}{2b^2}$ in $\frac{1}{2d^2}$. Od tod lahko izračunamo razdalje $|AB|$, $|AC|$ in $|BC|$.
Po konstrukciji je trikotnik $ABC$ pravokoten s pravim kotom v oglišču $C$, zato velja Pitagorov izrek
\[ |AB|^2 = |AC|^2 + |BC|^2. \]
Če dolžine izrazimo z $a, b, c$ in $d$, dobimo enakost
%
\begin{align*}
|AB|^2 
  &= \left (\frac{c}{d} - \frac{a}{b} \right )^2 + \left (\frac{1}{2d^2} - \frac{1}{2b^2} \right )^2 \\ 
  &= \left (\frac{bc-ad}{bd} \right )^2 + \frac{1}{4b^4} - \frac{1}{2b^2d^2} + \frac{1}{4d^4} \\
  &= \frac{(bc-ad)^2}{b^2d^2} + \left (\frac{1}{2b^2} + \frac{1}{2d^2} \right )^2 - \frac{1}{b^2d^2} \\
  &= \frac{(bc-ad)^2-1}{b^2d^2} + (|AD|^2 + |EB|^2).
\end{align*}

Če je $|bc-ad|>1$, je $|AB|^2 > |AD|^2 + |EB|^2$ in Fordova kroga C($\frac{a}{b}$) in C($\frac{c}{d}$) sta disjunktna.

Če je $|bc-ad|=1$, je $|AB|^2 = |AD|^2 + |EB|^2$ in Fordova kroga C($\frac{a}{b}$) in C($\frac{c}{d}$) sta tangentna.

Če je $|bc-ad|<1$, je $|bc-ad|=0,$ saj smo v množici celih števil. Sledi $\frac{a}{b} = \frac{c}{d},$ kar vodi v protislovje s predpostavko trditve.
\end{dokaz}

%%%%%%%%%%%%%%%%%%%%%%%%%%%%%%%%%%%%%%%%%%%%%%%%%%%%%%%%%%%%%%%%%%%%%%%%%
% Fordovi sosedi
\subsection{Fordovi sosedi}
%
Za tangentne Fordove kroge veljata naslednji lastnosti, ki lastnosti Fareyevih sosedov preneseta v jezik geometrije.

\begin{trditev}
\label{trd:FordTangentnost}
Fordova kroga C($\frac{a}{b}$) in C($\frac{c}{d}$) sta tangentna natanko tedaj, ko velja \( |bc-ad|=1. \)
\end{trditev}

\begin{dokaz}
%
Ponovno vpeljimo točke na Fordovih krogih kot v dokazu trditve~\ref{trd:FordDisjTang}. Implikacijo v levo smo že izpeljali, zato si oglejmo še implikacijo v desno.

Denimo, da sta Fordova kroga C($\frac{a}{b}$) in C($\frac{c}{d}$) tangentna. Potem za pravokotni trikotnik, ki ga določata, velja Pitagorov izrek 
\[ |AC|^2 + |BC|^2 = |AB|^2 \] oziroma
\[ \left (\frac{c}{d} - \frac{a}{b} \right )^2 + \left (\frac{1}{2b^2} - \frac{1}{2d^2} \right )^2 = \left (\frac{1}{2b^2} + \frac{1}{2d^2} \right )^2. \]
Ko odpravimo oklepaje, opazimo, da se nekateri členi odštejejo. Nato odpravimo ulomke in dobljeno enakost poenostavimo.
\[  \frac{c^2}{d^2} - \frac{2ac}{bd} + \frac{a^2}{b^2} + \frac{1}{4b^4} - \frac{1}{2b^{2}d^{2}} + \frac{1}{4d^4} = \frac{1}{4b^4} + \frac{1}{2b^{2}d^{2}} + \frac{1}{4d^4}, \]
\[ b^{2}c^{2} - 2abcd + a^{2}d^{2} - 1 = 0, \]
\[  (bc-ad)^2 = 1, \]
\[ |bc-ad|=1. \]
Trditev je s tem dokazana.
\end{dokaz}

\begin{definicija}
Tangentna Fordova kroga imenujemo \emph{Fordova soseda}.
\end{definicija}

\begin{trditev}
Naj bosta C($\frac{a}{b}$) in C($\frac{c}{d}$) Fordova soseda. Tedaj obstaja enolično določen Fordov krog C($\frac{a+c}{b+d}$) in je tangenten na izbrana kroga.
\end{trditev}

\begin{dokaz}
Po definiciji Fordovih krogov vemo, da sta $\frac{a}{b}$ in $\frac{c}{d}$ okrajšana ulomka in zaradi tangentnosti Fareyeva soseda v nekem Fareyevem zaporedju (razširjenem na celotno realno os). Po lemi~\ref{lema:MediantaOkrUlom} je njuna medianta $\frac{a+c}{b+d}$ tudi okrajšan ulomek, torej obstaja natanko en Fordov krog C($\frac{a+c}{b+d}$). 

Dokažimo še, da je C($\frac{a+c}{b+d}$) tangenten na izbrana kroga. Ker sta C($\frac{a}{b}$) in C($\frac{c}{d}$) Fordova soseda, velja zveza 
\( |bc-ad|=1. \)
Če jo nekoliko preoblikujemo, dobimo
\[ |bc-ad|=|bc-ad+cd-cd|=|(b+d)c-(a+c)d|=1, \]
od koder sledi, da sta Fordova kroga C($\frac{a+c}{b+d}$) in C($\frac{c}{d}$) tangentna.
Podobno
\[ |bc-ad|=|bc-ad+ab-ab|=|(a+c)b-(b+d)a|=1 \]
pomeni, da sta Fordova kroga C($\frac{a}{b}$) in C($\frac{a+c}{b+d}$) tangentna.
\end{dokaz}

% konstrukcija Fordovih sosedov
Naslednji izrek pove, kako konstruiramo množico vseh Fordovih sosedov danega Fordovega kroga.

\begin{izrek}
Naj bosta kroga C($\frac{p}{q}$) in C($\frac{P}{Q}$) Fordova soseda. Vse Fordove sosede Fordovega kroga C($\frac{p}{q}$) lahko zapišemo v obliki C($\frac{P_n}{Q_n}$), kjer je $\frac{P_n}{Q_n} = \frac{P+np}{Q+nq}$ in $n$ preteče vsa cela števila.
\end{izrek}

\begin{dokaz}
%
Najprej dokažimo, da sta Fordova kroga C($\frac{p}{q}$) in C($\frac{P_n}{Q_n}$) res Fordova soseda. Računajmo
\[ |qP_{n} - pQ_{n}| = |q(P+np) - p(Q+nq)| = |qP+qnp-pQ-pnq| = |qP-pQ| = 1. \]
Zadnja enakost velja po predpostavki, da sta C($\frac{p}{q}$) in C($\frac{P}{Q}$) Fordova soseda.

Sedaj preverimo, če obstajajo še Fordovi sosedi, ki niso zgornje oblike. Opazovali bomo zaporedje Fordovih krogov $\mathcal{M} = \{ \mathrm{C}(\frac{P_n}{Q_n}); n\in\Z \}.$
Iz računa
\begin{align*}
|Q_{n}P_{n+1} - P_{n}Q_{n+1}| 
  &= |(Q+nq)(P+(n+1)p) - (P+np)(Q+(n+1)q)| \\ 
  &= |QP + (n+1)Qp + nPq + n(n+1)pq \\
  &   - PQ - (n+1)Pq - npQ - n(n+1)pq| \\
  &= |Qp - pQ| \\
  &= 1
\end{align*}
sledi, da sta zaporedna elementa zaporedja $\mathcal{M}$ Fordova soseda.
Ulomek $\frac{P_n}{Q_n}$, ki predstavlja Fordov krog C($\frac{P_n}{Q_n}$), lahko zapišemo kot
\begin{align*}
\frac{P_n}{Q_n}
  &= \frac{P+np}{Q+nq} = \frac{Pq+npq}{q(Q+nq)} = \frac{Pq+npq+pQ-pQ}{q(Q+nq)} = \frac{p(Q+nq) + (Pq-pQ)}{q(Q+nq)} \\
  &= \frac{p}{q} + \frac{Pq-pQ}{q(Q+nq)} = \frac{p}{q} \pm \frac{1}{q(Q+nq)} = \frac{p}{q} \pm \frac{1}{q^2 \left (n+\frac{Q}{q} \right)}.
\end{align*}
V limiti, ko gre $n$ preko vseh meja, gre $\frac{P_n}{Q_n}$ proti $\frac{p}{q}.$
Ugotovili smo, da Fordovi krogi oblike C($\frac{P_n}{Q_n}$) geometrijsko tvorijo obroč okroli Fordovega kroga C($\frac{p}{q}$). Z njim so namreč vsi tangentni, prav tako pa so tangentni tudi na svojega predhodnika in naslednika v zaporedju $\mathcal{M}$. Njihova dotikališča z abscisno osjo konvergirajo proti točki $\frac{p}{q},$ ki je dotikališče danega Fordovega kroga C($\frac{p}{q}$), zaradi medsebojne tangentnosti pa so njihovi polmeri vse manjši. Zato ne obstaja Fordov krog, tangenten na C($\frac{p}{q}$), ki ni zgornje oblike in ne seka katerega izmed krogov iz zaporedja $\mathcal{M}$.
%
\end{dokaz}

% Pitagorejske trojice
%
V dokazu trditve~\ref{trd:FordDisjTang} smo konstruirali pravokotni trikotnik, določen s središčema tangentnih Fordovih krogov in presečiščem premic skozi središči. Spomnimo se znane definicije iz teorije števil, ki izhaja iz evklidske geometrije. 

\begin{definicija}
Trojica naravnih števil $(a,b,c)$, za katero velja $a^2+b^2=c^2$, se imenuje \emph{pitagorejska trojica}\footnote{Pojem pitagorejska trojica nosi ime slavnega starogrškega matematika Pitagore (okoli 570~pr.~n.~št.--495~pr.~n.~št.), ki ga poznamo predvsem po Pitagorovem izreku.}.
Pitagorejska trojica je \emph{primitivna}, če števila $a$, $b$, in $c$ nimajo skupnega faktorja.
\end{definicija}

\begin{trditev}
Pravokotna trikotnika, ki pripadata poljubnima paroma Fordovih sosedov, določata različni primitivni pitagorejski trojici.
\end{trditev}

\begin{dokaz}
%
Naj bosta C($\frac{a}{b}$) in C($\frac{c}{d}$) ter C($\frac{a'}{b'}$) in C($\frac{c'}{d'}$) poljubna različna para Fordovih sosedov. Brez škode za splošnost naj velja $\frac{a}{b}<\frac{c}{d}$ in $\frac{a'}{b'}<\frac{c'}{d'}$. Naj prvemu paru Fordovih sosedov pripada pravokotni trikotnik $ABC$, drugemu paru pa pravokotni trikotnik $A'B'C'$. Dokazati želimo, da si trikotnika nista podobna.

Pa denimo, da sta si trikotnika $ABC$ in $A'B'C'$ podobna. Tedaj obstaja tako naravno število $\lambda\ne{1}$, da za dolžine stranic obeh pravokotnih trikotnikov veljajo naslednje zveze:
%
\begin{align}
\frac{1}{2b^2}-\frac{1}{2d^2} &= \lambda \left (\frac{1}{2b'^2}-\frac{1}{2d'^2} \right ), \\
\frac{1}{2b^2}+\frac{1}{2d^2} &= \lambda \left (\frac{1}{2b'^2}+\frac{1}{2d'^2} \right ), \\
\frac{c}{d}-\frac{a}{b} &= \lambda \left (\frac{c'}{d'}-\frac{a'}{b'} \right ).
\end{align}

Če seštejemo prvi dve enačbi, dobimo
%
\begin{align}
\frac{1}{b^2} &= \lambda \frac{1}{b'^2}, \nonumber \\
b'^2 &= \lambda b^2, \nonumber \\
b' &= \sqrt\lambda b.
\end{align}

Enačbo (3) lahko poenostavimo, saj gre za para Fordovih sosedov. Velja
\begin{equation}
\frac{1}{bd} = \frac{bc-ad}{bd} = \frac{c}{d}-\frac{a}{b} = \lambda \left (\frac{c'}{d'}-\frac{a'}{b'} \right ) = \lambda \frac{b'c'-a'd'}{b'd'} = \lambda \frac{1}{b'd'}.
\end{equation}

Iz (4) in (5) sledi 
%
\begin{align}
\frac{1}{bd} &=\lambda \frac{1}{\sqrt\lambda bd'}, \nonumber \\
\frac{1}{d} &= \frac{\sqrt\lambda}{d'}, \nonumber \\
d' &= \sqrt\lambda d.
\end{align}

Nazadnje še v pogoj za tangentnost Fordovih krogov C($\frac{a'}{b'}$) in C($\frac{c'}{d'}$) vstavimo zvezi (4) in (6):
%
\begin{align}
b'c'-a'd' &= 1, \nonumber \\
\sqrt\lambda bc' - a' \sqrt\lambda d &= 1, \nonumber \\
\sqrt\lambda (bc'-a'd) &= 1.
\end{align}

To pa je možno le tedaj, ko je $\lambda=1.$ Prispeli smo do protislovja, kar pomeni, da si trikotnika nista podobna.
%
\end{dokaz}

%%%%%%%%%%%%%%%%%%%%%%%%%%%%%%%%%%%%%%%%%%%%%%%%%%%%%%%%%%%%%%%%%%%%%%%%%
% Posplositve
\subsection{Posplošeni Fordovi krogi}

V prejšnjem razdelku smo se ukvarjali s Fordovimi krogi, ki so bili enolično določeni z racionalnim številom. Natančneje, za dan okrajšan ulomek $\frac{p}{q}$ smo konstruirali Fordov korg na zgornji polravnini evklidkse ravnine, ki se abscisne osi dotika v točki $\frac{p}{q}$, njegov polmer pa meri $\frac{1}{2q^2}.$
Nadaljujemo lahko s splošnejšimi Fordovimi krogi, ki so definirani na povsem enak način, le da imajo polmer enak $\frac{1}{2hq^2},$ kjer je $h$ poljubno realno število. Imenujemo jih tudi Speiserjevi\footnote{Andreas Speiser, 10.~6.~1885--12.~10.~1970, švicarski matematik, ki se je ukvarjal s teorijo števil in teorijo grup.} krogi. Če izberemo $h=1,$ dobimo običajne Fordove kroge.

%%%%%%%%%%%%%%%%%%%%%%%%%%%%%%%%%%%%%%%%%%%%%%%%%%%%%%%%%%%%%%%%%%%%%%%%%
% Fordove krogle
\subsection{Fordove krogle}

Naj bosta sedaj števili $p$ in $q$ elementa množice Gaussovih celih števil, ki je definirana kot $\mathbb{Z}\left[{i}\right] = \{a+bi; a,b \in \Z \}.$ Zapišimo $p = p'+ip''$ in $q = q'+iq'',$ kjer so $p', p'', q', q'' \in \Z.$ Definirajmo ulomek 
\[ \frac{p}{q} = \frac{p'+ip''}{q'+iq''} = \frac{(p'+ip'')(q'-iq'')}{(q'+iq'')(q'-iq'')} = \frac{p'q'+p''q''}{q'^2+q''^2} + i \frac{p''q'-p'q''}{q'^2+q''^2}, \]
ki pripada kompleksnim številom, in ga okrajšajmo. Geometrijsko predstavlja točko v Gaussovi $xy$-ravnini z realno in imaginarno koordinato. Gaussovo ravnino postavimo v prostor, določen z osjo $z$, pravokotno na Gaussovo ravnino, in opazujmo podprostor, ki pripada pozitivnim vrednostim na $z$-osi. Analogno Fordovm krogom v ravnini pridemo do naslednjega pojma. 

\begin{definicija}
\emph{Fordova krogla} S($\frac{p}{q}$), kjer je $\frac{p}{q}$ okrajšan ulomek v množici kompleksnih števil, je krogla v zgornjem polprostoru, definiranem kot zgoraj, ki se $xy$-ravnine dotika v točki, določeni s $\frac{p}{q}$, njen polmer pa meri $\frac{1}{2|q|^2}.$
\end{definicija}

% lastnosti
Na analogen način trditvam, ki opisujejo lastnosti Fordovih krogov, lahko izpeljemo in dokažemo lastnosti Fordovih krogel. Omenimo le nekatere izmed njih.

Poljubno majhen zaprt pravokotnik v $xy$-ravnini vsebuje neskončno mnogo dotikališč Fordovih krogel.

V dokazu trditve \ref{trd:FordDisjTang} smo konstruirali pravokotni trikotnik, določen z dvema Fordovima krogoma. Naj točki $\frac{p}{q}$ in $\frac{P}{Q}$ določata Fordovi krogli ter konstruirajmo pravokotni trikotnik $ABC$ kot prej. Iz zveze
\begin{align*}
|AB|^2 
  &= \left |\frac{P}{Q} - \frac{p}{q} \right|^2 + \left |\frac{1}{2|Q|^2} - \frac{1}{2|q|^2} \right|^2 = \frac{|Pq-pQ|^2-1}{|Q|^2|q|^2} + |AD|^2 + |EB|^2
\end{align*}
sledi:
če je $|Pq-pQ|>1,$ je $|AB|^2>|AD|^2+|EB|^2$ in krogli sta disjunktni;
sicer je $|Pq-pQ|=1,$ zato je $|AB|^2=|AD|^2+|EB|^2$ in krogli sta tangentni.

Naj bosta S($\frac{p}{q}$) in S($\frac{P}{Q}$) tangentni Fordovi krogli. Kot prej tangentne Fordove krogle na kroglo S($\frac{p}{q}$) konstruiramo s pomočjo formule
\[ \frac{P_n}{Q_n} = \frac{P+np}{Q+nq}, \]
le da tokrat $n$ pripada množici Gaussovih celih števil. 
Nadalje nas zanima, koliko Fordovih krogel je tangentnih na kroglo S($\frac{P_n}{Q_n}$).
Uporabimo pogoj za tangentnost, izpeljan zgoraj. Računamo
\begin{align*} 
|P_{n}Q_{m} - P_{m}Q_{n}| 
  &= |(P+np)(Q+mq) - (P+mp)(Q+nq)| \\
  &= |PQ+mPq+npQ+mnpq-PQ-nqP-mpQ-mnpq| \\
  &= |Pq-pQ| |m-n| = 1.
\end{align*}
%
Sledi, da je $|m-n|=1,$ torej je $m-n \in \{1, -1, i, -i \}.$ Ugotovili smo, da je vsaka Fordova krogla, ki je tangentna na dano Fordovo kroglo, tangentna še na štiri druge Fordove krogle. Velja, da so te edine tangentne krogle.

%%%%%%%%%%%%%%%%%%%%%%%%%%%%%%%%%%%%%%%%%%%%%%%%%%%%%%%%%%%%%%%%%%%%%%%%%% Mobiusove transformacije na Fordovih krogih
\subsection{M\"{o}biusove transformacije na Fordovih krogih}

Do sedaj smo Fordove kroge obravnavali s pomočjo geometrijskih sredstev. V tem poglavju bomo preko algebre prišli do znanih rezultatov o Fordovih krogih.
Geometrijske objekte si bomo predstavljali v kompleksni ravnini, torej bo točka s koordinatama $(x,y)$ opisana s kompleksnim številom $z=x+iy$.

% Mobiusova transformacija
Najprej se spomnimo naslednjega pojma iz kompleksne analize.
\begin{definicija}
Preslikava \( f \colon \mathbb{CP}^{1} \rightarrow \mathbb{CP}^{1} \), definirana s predpisom \( f(z) = \frac{az+c}{bz+d} \), kjer so $a,b,c,d \in \mathbb{C}$ in $ad-bc \neq 0$, se imenuje \emph{M\"{o}biusova transformacija}.
\end{definicija} 

\begin{opomba}
Simbol $ \mathbb{CP}^{1}$ označuje \emph{Riemannovo sfero}, to je kompaktifikacijo kompleksne ravnine z eno točko, kar zapišemo kot $\mathbb{CP}^{1} = \mathbb{C} \cup \{\infty\}$.
\end{opomba}

\begin{opomba}
Števila $a,b,c,d$ lahko pomnožimo s poljubnim neničelnim kompleksnim številom, zato lahko predpostavimo, da je $ad-bc=1$.
Dodatno, v našem primeru bo dovolj obravnavati le $a,b,c,d \in \mathbb{Z}$.
\end{opomba}

M\"{o}biusova transformacija je meromorfna in bijektivna preslikava, identična preslikava $id(z)=z$ je M\"{o}biusova transformacija, prav tako je kompozitum M\"{o}biusovih transformacij spet M\"{o}biusova transformacija. Množica takih preslikav tako tvori grupo za kompozitum. 
Preslikavo $f$ lahko zapišemo v matrični obliki
\[
\mathbf{A}\ =
\left[
\begin{array}{cc}
a & c \\
b & d \\
\end{array}
\right],
\]
kjer so $a,b,c,d \in \mathbb{Z}$. Ker velja $ad-bc=1$, je matrika $A \in {SL}_{2}(\mathbb{Z})$. Zato bomo M\"{o}biusove transformacije predstavljali s splošno linearno grupo ${SL}_{2}(\mathbb{Z})$.

% Delovanje grupe na množico
Ključen pri izpeljavi rezultatov o Fordovih krogih je pojem, ki ga v algebri pogosto uporabljamo.

\begin{definicija}
\emph{Delovanje grupe} $G$ na množico $M$ je taka preslikava \( \circ \colon G \times M \rightarrow M \), za katero velja:
\begin{enumerate}
	\item \( e \circ \alpha = \alpha \) za vsak $ \alpha \in M$, kjer je $e$ enota grupe $G$, 
	\item \( g \circ (h \circ \alpha) = (gh) \circ \alpha \) za vsak $\alpha \in M$ in vsaka $g,h \in G$.
\end{enumerate}

Ekvivalentno, delovanje grupe $G$ na množico $M$ je homomorfizem iz grupe $G$ v grupo permutacij množice $M$.
\end{definicija}

\begin{primer}
Naj bo $G$ grupa permutacij $n$ elementov, torej $G = S_{n}$, $M$ pa naj bo množica $M = \{1,2, \ldots, n \}$.
Delovanje grupe $G$ na množico $M$ je preslikava \( \circ \colon G \times M \rightarrow M \) s predpisom \( (\pi, \alpha) \mapsto \pi(\alpha) = \pi \circ \alpha \).
\end{primer}

% Fordovi krogi
\begin{definicija}
Fordov krog C($\frac{1}{0}$), katerega polmer je neskončen, je premica $\mathbb{R} + i$.
\end{definicija}

\begin{izrek}
M\"{o}biusova transformacija slika Fordove kroge v Fordove kroge.
\end{izrek}

\begin{dokaz}
Naj bo M\"{o}biusova transformacija dana z matriko 
\[
\mathbf{A}\ =
\left[
\begin{array}{cc}
a & c \\
b & d \\
\end{array}
\right]
\in {SL}_{2}(\mathbb{Z}).
\]
Vemo , da M\"{o}biusova transformacija slika premice in krožnice v premice in krožnice. Da bomo dokazali, da Fordove kroge preslika v Fordove kroge, bomo pokazali, da grupa ${SL}_{2}(\mathbb{Z})$ deluje na množico Fordovih krogov.
S krajšim računom lahko preverimo, da je grupa ${SL}_{2}(\mathbb{Z})$ generirana z matrikama 
\(
\left[
\begin{array}{cc}
1 & 1 \\
0 & 1 \\
\end{array}
\right]
\)
in
\(
\left[
\begin{array}{cc}
0 & -1 \\
1 & 0 \\
\end{array}
\right]
\).
To pomeni, da je vsako M\"{o}biusovo transformacijo moč zapisati kot kompozitum preslikav $z \mapsto z+1$ in $z \mapsto -\frac{1}{z}$, ki ustrezata generatorjema. Trdimo, da M\"{o}biusova transformacija preslika Fordov krog z intervala $ [0,1]$ v Fordov krog z intervala $[n,n+1]$, kjer $n$ pripada množici celih števil. 

Preslikava $z \mapsto z+1$ je translacija, zato zgornje očitno velja.

Oglejmo si še preslikavo $z \mapsto -\frac{1}{z}$. Naj bo original Fordov krog s polmerom $r$ in središčem v $\alpha$, predstavljen z enačbo $ |z-\alpha|=r$. Prepišimo enačbo v
\begin{align}
(z-\alpha)( \bar{z}-\bar{\alpha}) &= r^2, \nonumber \\
z\bar{z} - \alpha \bar{z} - \bar{\alpha} z + \alpha \bar{\alpha} - r^2 &= 0
\end{align}
in označimo $R = \alpha \bar{\alpha} - r^2$. Polmer lahko izrazimo kot $r = \sqrt{|\alpha|^2-R}$.
Preslikajmo original s preslikavo $z \mapsto -\frac{1}{z}$. Enačba slike se glasi
 \begin{align}
\left(-\frac{1}{z}\right)\left(-\frac{1}{\bar{z}}\right) + \alpha \frac{1}{\bar{z}} + \bar{\alpha} \frac{1}{z} + R &= 0, \nonumber \\
1 + \alpha z + \bar{\alpha} \bar{z} + Rz \bar{z} &= 0, \nonumber \\
z \bar{z} + \frac{\bar{\alpha}}{R} \bar{z} + \frac{\alpha}{R} z + \frac{1}{R} &= 0.
\end{align}

\end{dokaz}

%
%%%%%%%%%%%%%%%%%%%%%%%%%%%%%%%%%%%%%%%%%%%%%%%%%%%%%%%%%%%%%%%%%%%%%%%%%%%%%%%%%%%%%%%%%%%%%%%%%%%%%%%%%%%%%%%%%%%%%%%%%%%%%%%%%%%%%%%%%%%%%%%%%%


\section*{Slovar strokovnih izrazov}

\geslo{}{}
\geslo{}{}

% seznam uporabljene literature
\begin{thebibliography}{99}

\bibitem{fareyproject} J.~Ainsworth, M.~Dawson, J.~Pianta in J.~Warwick, \emph{The Farey sequence}, diplomsko delo, School of Mathematics, University of Edinburgh, 2012; dostopno tudi na \url{https://www.maths.ed.ac.uk/~v1ranick/fareyproject.pdf}.

\bibitem{ford} L.~R.~Ford, \emph{Fractions}, v: The American Mathematical Monthly (ur.\ E.~J.~Moulton) \textbf{45}, Mathematical Association of America, 1938, str.\ 586--601.

\bibitem{motifofmath} S.~B.~Guthery, \emph{A motif of mathematics}, Docent Press, Boston, 2011; dostopno tudi na \url{https://www.maths.ed.ac.uk/~v1ranick/papers/farey.pdf}.

\bibitem{hardy} G.~H.~Hardy in E.~M.~Wright, \emph{An introduction to the theory of numbers}, 4th ed., Oxford University Press, Oxford, 1960.

\bibitem{TNBook} A.~Hatcher, \emph{Topology of numbers}, verzija junij 2018, [ogled 31.~10.~2018], dostopno na \url{https://pi.math.cornell.edu/~hatcher/TN/TNpage.html}.

\bibitem{strnad} M.~Strnad, \emph{Pitagorov izrek pred Pitagorom}, Presek \textbf{17} (1989) 8--11; dostopno tudi na \url{http://www.presek.si/17/966-Strnad.pdf}.

\end{thebibliography}

\end{document}

