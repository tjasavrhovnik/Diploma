\documentclass[mat1]{fmfdelo}
% \documentclass[fin1]{fmfdelo}
% \documentclass[isrm1]{fmfdelo}
% \documentclass[mat2]{fmfdelo}
% \documentclass[fin2]{fmfdelo}
% \documentclass[isrm2]{fmfdelo}

% naslednje ukaze ustrezno napolnite
\avtor{Tjaša Vrhovnik}

\naslov{Fareyevo zaporedje in Riemannova hipoteza}
\title{The Farey Sequence and The Riemann Hypothesis}

% navedite ime mentorja s polnim nazivom: doc.~dr.~Ime Priimek,
% izr.~prof.~dr.~Ime Priimek, prof.~dr.~Ime Priimek
% uporabite le tisti ukaz/ukaze, ki je/so za vas ustrezni
\mentor{izr.~prof.~dr.~Aleš Vavpetič}
% \mentorica{}
% \somentor{}
% \somentorica{}
% \mentorja{}{}
% \mentorici{}{}

\letnica{2019} % leto diplome

%  V povzetku na kratko opišite vsebinske rezultate dela. Sem ne sodi razlaga organizacije dela --
%  v katerem poglavju/razdelku je kaj, pač pa le opis vsebine.
\povzetek{}

%  Prevod slovenskega povzetka v angleščino.
\abstract{}

% navedite vsaj eno klasifikacijsko oznako --
% dostopne so na www.ams.org/mathscinet/msc/msc2010.html
\klasifikacija{}
\kljucnebesede{} % navedite nekaj ključnih pojmov, ki nastopajo v delu
\keywords{} % angleški prevod ključnih besed

\zapisiMetaPodatke  % poskrbi za metapodatke in veljaven PDF/A-1b standard

% aktivirajte pakete, ki jih potrebujete
% \usepackage{tikz}

% za številske množice uporabite naslednje simbole
\newcommand{\R}{\mathbb R}
\newcommand{\N}{\mathbb N}
\newcommand{\Z}{\mathbb Z}
\newcommand{\C}{\mathbb C}
\newcommand{\Q}{\mathbb Q}

% matematične operatorje deklarirajte kot take, da jih bo Latex pravilno stavil
% \DeclareMathOperator{\conv}{conv}

% vstavite svoje definicije ...
%  \newcommand{}{}

\begin{document}

%%%%%%%%%%%%%%%%%%%%%%%%%%%%%%%%%%%%%%%%%%%%%%%%%%%%%%%%%%%%%%%%%%%%%%%%%%%%%%%%%%%%%%%%%%%%%%%%%%%%%%%%%%%%%%%%%%%%%%%%%%%%%%%%%%%%%%%%%%%%%%%%%%
%
\section{Uvod}

Zgodovina Fareyevega zaporedja sega v London 18.~stoletja. Med letoma 1704 in 1841 je izhajal letni zbornik \emph{The Ladies Diary: or, the Woman's Almanack}, ki je povezoval ljubitelje matematičnih ugank. Bralce so namreč nagovarjali k pošiljanju in reševanju aritmetičnih problemov, ki so bili v zborniku objavljeni. Leta~1747 se je pojavilo naslednje vprašanje: Najti je potrebno število ulomkov različnih vrednosti, manjših od $1$, katerih imenovalec ni večji od $100$. Odziv je bil precejšen, saj so se z iskanjem rešitve ukvarjali tako javnost kot pomembni matematiki tiste dobe. To je vodilo v razvoj Fareyevega zaporedja, ki ima nekaj presenetljivih lastnosti in uporabo na različnih področjih matematike.

Delo diplomskega seminarja je sestavljeno iz treh večjih enot. V prvi bomo predstavili zgodovinski pregled, definicijo, lastnosti in ocenili dolžino Fareyevega zaporedja. V drugem razdelku se bomo posvetili geometrijski interpretaciji zaporedja -- Fordovim krogom -- in opazili, da imajo lastnosti zaporedja tudi geometrijski pomen. Navedli bomo nekaj posplošitev Fordovih krogov ter si ogledali njihovo predstavitev z uporabo algebre in kompleksne analize. V zadnjem delu bomo obravnavali znamenit matematični problem, Riemannovo hipotezo. Opisali bomo povezavo Riemannove hipoteze s Fareyevim zaporedjem in dokazali njeno ekvivalenčno trditev. 

%%%%%%%%%%%%%%%%%%%%%%%%%%%%%%%%%%%%%%%%%%%%%%%%%%%%%%%%%%%%%%%%%%%%%%%%%%%%%%%%%%%%%%%%%%%%%%%%%%%%%%%%%%%%%%%%%%%%%%%%%%%%%%%%%%%%%%%%%%%%%%%%%%
%
\section{Fareyevo zaporedje}

%%%%%%%%%%%%%%%%%%%%%%%%%%%%%%%%%%%%%%%%%%%%%%%%%%%%%%%%%%%%%%%%%%%%%%%%%
%
\subsection{Zgodovina Fareyevega zaporedja}

Vrnimo se k v uvodu omenjeni nalogi o številu ulomkov različnih vrednosti, manjših od $1$, z imenovalci kvečjemu $100$. Prvi odgovor na članek je bila tabela ulomkov z imenovalci manjšimi od $10$, nato pa še dve tabeli z rezultatoma $3055$ in $4851$. Leta 1751 je R.~Flitcon objavil pravilni odgovor $3003$, kateremu je dodal tudi opis postopka. Preden ga razložimo, si oglejmo Eulerjevo funkcijo in njene lastnosti, ki nas bo pripeljala do rešitve. 
%
% Eulerjeva funkcija

\begin{definicija}
Preslikava \( \varphi \colon \mathbb{N} \rightarrow \mathbb{N}\), ki za vsako naravno število $n$ prešteje števila, manjša od $n$, ki so $n$ tuja, se imenuje \emph{Eulerjeva funkcija} $\varphi$.
\end{definicija}

\begin{trditev}
\label{trd:MultipEuler}
Če sta $k$ in $l$ tuji si števili, velja $\varphi (kl) = \varphi (k) \varphi (l)$, torej je Eulerjeva funkcija multiplikativna.
\end{trditev}

\begin{dokaz}
V dokazu multiplikativnosti si bomo pomagali z lastnostmi grup. 
Naj $\mathbb{Z}_{k}^\ast $ označuje grupo vseh obrnljivih elementov grupe $\mathbb{Z}_{k}$. Vemo, da so obrnljivi elementi grupe $\mathbb{Z}_{k}$ tista števila iz množice \( \{0, 1, \ldots, k-1 \}, \) ki so tuja $k$, zato je $|\mathbb{Z}_{k}^\ast| = \varphi(k).$  
Dobimo zvezi
\[ |\mathbb{Z}_{kl}^\ast| = \varphi(kl), \]
\[ |\mathbb{Z}_{k}^\ast| |\mathbb{Z}_{l}^\ast| = \varphi(k) \varphi(l).\]
Znano je, da je preslikava \( \psi \colon \mathbb{Z}_{kl}^\ast \rightarrow \mathbb{Z}_{k}^\ast \times \mathbb{Z}_{l}^\ast \) za tuji naravni števili $k$ in $l$ izomorfizem grup. Ker je moč kartezičnega produkta dveh množic enaka produktu njunih moči, sledi 
\[ \varphi(kl) = |\mathbb{Z}_{kl}^\ast| = |\mathbb{Z}_{k}^\ast| |\mathbb{Z}_{l}^\ast| = \varphi(k) \varphi(l), \]
s čimer je multiplikativnost dokazana.
\end{dokaz}

\begin{trditev}
\label{trd:EulerPrastProd}
Vrednost Eulerjeve funkcije je enaka
\[ \varphi(n) = n \prod_{p|n} \left (1 - \frac{1}{p} \right ), \] kjer so $p$ prafaktorji števila $n$.
\end{trditev}

\begin{dokaz}
Zapišimo $n$ kot produkt prafaktorjev, \(n=\prod_{i=1}^m p_i^{r_i}\), kjer so $r_i \in \mathbb{N}$ in $p_i$ praštevila.
Funkcija $\varphi(p^{r})$ prešteje vsa števila, manjša od $p^r$, ki so tuja $p^r$. To so natanko tista, ki niso deljiva s praštevilom $p$. Večkratnikov $p$ med števili $1, 2, \ldots, p^r-1$ je toliko kot večkratnikov $p$ med števili $1, 2, \ldots, p^r$, teh pa je \( \frac{p^r}{p} = p^{r-1}. \)
Torej je 
\[ \varphi(p^r) = (p^r - 1) - (p^{r-1} - 1) = p^r - p^{r-1} = p^r \left (1 - \frac{1}{p} \right). \] 
Z upoštevanjem multiplikativnosti funkcije $\varphi$ dobimo
%
\begin{align*}
 \varphi(n) 
 &= \varphi \left (\prod_{i=1}^m p_i^{r_i} \right ) = \prod_{i=1}^m \varphi (p_i^{r_i} ) = 
 \prod_{i=1}^m p_i^{r_i} \left (1 - \frac{1}{p_i} \right ) \\
 &= \prod_{i=1}^m p_i^{r_i} \times \prod_{i=1}^m \left (1 - \frac{1}{p_i} \right ) = 
 n \prod_{i=1}^m \left (1 - \frac{1}{p_i} \right ) = n \prod_{p|n} \left (1 - \frac{1}{p} \right ),
\end{align*}
kar smo želeli dokazati.
\end{dokaz}
%
% Flitconova rešitev

\begin{trditev}
Obstajajo $3003$ racionalna števila $\frac{p}{q}$, za katera velja $0<\frac{p}{q}<1$ ter je $q \leq 100$.
\end{trditev}

Namesto formalnega dokaza trditve bomo predstavili Flitconovo rešitev. Sledili bomo \cite[poglavje 1.2]{fareyproject}. Naredimo tabelo s tremi stolpci in $99$ vrsticami. V prvi stolpec vsake vrstice napišemo po eno izmed naravnih števil od $2$ do $100$. V drugi stolpec posamezne vrstice zapišemo naravno število iz prvega stolpca kot produkt prafaktorjev, v tretji stolpec pa vrednost $\varphi(n)$. Pomagamo si s trditvama~\ref{trd:MultipEuler} in \ref{trd:EulerPrastProd}. Vsota vrednosti v tretjem stolpcu nam da število iskanih ulomkov. Res, vsak $\varphi(n)$ nam pove število okrajšanih ulomkov med $0$ in $1$ z imenovalcem $n$, vsota vrednosti Eulerjeve funkcije $\varphi(n)$ za vsa števila $n$ med $2$ in $100$ pa število vseh okrajšanih ulomkov med $0$ in $1$ z imenovalci med $2$ in $100$.\footnote{Čeprav Flitcon ne omenja Eulerjeve funkcije $\varphi$, je uporabil njene lastnosti v svoji matematično manj formalni metodi.}

Neodvisno od Flitconove rešitve je francoski matematik Charles Haros leta 1802 sestavil enak seznam ulomkov, vendar na precej bolj zanesljiv način. Haros se dela ni lotil z željo po reševanju aritmetične naloge, pač pa je pisal tabele za pretvarjanje med ulomki in decimalnim zapisom ter obratno. V Franciji so namreč v času revolucije konec 18.~stoletja uvajali nov metrični sistem, ki je med drugim zahteval uporabo decimalnega zapisa. Tabele so bile objavljene v časniku \emph{Journal de l'Ecole Polytechnique}, primerom ter algoritmom za pretvarjanje pa so bile dodane skice dokazov in nekatere lastnosti zaporedja ulomkov, ki so kasneje postali znani pod imenom Fareyevo zaporedje. 

Posebej zanimiva je zgodba o pivovarju in ljubiteljskemu matematiku Henryju Goodwynu. Čeprav ni imel formalne izobrazbe, se je navduševal nad znanostjo in tehniko, sestavljal različne tabele in računal, kako izboljšati svoje poslovanje. Po upokojitvi se je vse bolj posvečal matematiki -- tako je med letoma 1816 in 1823 objavil več člankov s tabelami okrajšanih ulomkov. Njegovo delo sta opazila znameniti francoski matematik Augustin Louis Cauchy in John Farey, geolog, po komer se obravnavano zaporedje okrajšanih ulomkov imenuje. Vemo, da je Cauchy prispeval nekaj dokazov lastnosti Fareyevega zaporedja, v nasprotju pa ostaja neznano, ali sta Goodwyn in Farey zaporedje in nekatere njegove lastnosti odkrila neodvisno od Harosa, bodisi sta vedela za njegove ugotovitve. Farey je najverjetneje na podlagi Goodwynovih tabel maja 1816 v pismu časopisu \emph{The Philosophical Magazine and Journal} z naslovom \emph{On a curious Property of vulgar Fractions} predstavil medianto, najpomembnejšo lastnost zaporedja. Čeprav zaporedje morda neupravičeno nosi ime Johna Fareya, pa ne smemo spregledati njegovega prispevka k raziskovanju matematike v glasbi, vzorcev, astronomije in seveda geologije. Zgodovina je povzeta po \cite[poglavje 2]{motifofmath}.

%%%%%%%%%%%%%%%%%%%%%%%%%%%%%%%%%%%%%%%%%%%%%%%%%%%%%%%%%%%%%%%%%%%%%%%%%
% Fareyevo zaporedje
%
\subsection{O Fareyevem zaporedju}

Motivacijo za razvoj Fareyevega zaporedja smo si ogledali v prejšnjem razdelku. Sedaj bomo zaporedje korektno definirali in izpeljali njegove lastnosti.

\begin{definicija}
\label{def:Farey}
\emph{Fareyevo zaporedje reda n} oz.\ \emph{n-to Fareyevo zaporedje} je množica racionalnih števil $\frac{p}{q}$ urejenih po velikosti, kjer sta $p$ in $q$ tuji si števili, ter velja $0 \leq p \leq q \leq n$. Označimo ga z $F_n$.

Ekvivalentno, $F_n$ vsebuje vse okrajšane ulomke med $0$ in $1$ z imenovalci, kvečjemu enakimi $n$.
\end{definicija}

\begin{primer}
Poglejmo si prvih nekaj Fareyevih zapredij.

\(F_1 = \left \{\frac{0}{1}, \frac{1}{1} \right \}, \)

\(F_2 = \left \{\frac{0}{1}, \frac{1}{2}, \frac{1}{1} \right \}, \)

\(F_3 = \left \{\frac{0}{1}, \frac{1}{3}, \frac{1}{2}, \frac{2}{3}, \frac{1}{1} \right \}, \)

\(F_4 = \left \{\frac{0}{1}, \frac{1}{4}, \frac{1}{3}, \frac{1}{2}, \frac{2}{3}, \frac{3}{4}, \frac{1}{1} \right \}, \)

\(F_5 = \left \{\frac{0}{1}, \frac{1}{5}, \frac{1}{4}, \frac{1}{3}, \frac{2}{5}, \frac{1}{2}, \frac{3}{5}, \frac{2}{3}, \frac{3}{4}, \frac{4}{5}, \frac{1}{1}\right \}. \)
\end{primer}

\begin{opomba}
Če pogoj $0 \leq p \leq q \leq n$ v definiciji~\ref{def:Farey} omilimo v pogoj $0 \leq p,q \leq n$, okrajšane ulomke z intervala $[0,1]$ razširimo na interval $[0, \infty)$. V primeru, ko za $p$ in $q$ dovoljujemo tudi negativna cela števila, dobimo okrajšane ulomke na celotni realni osi.
\end{opomba}

V zgornjih primerih opazimo, da za vsaka sosednja člena Fareyevega zaporedja velja naslednje. Če števec prvega ulomka množimo z imenovalcem drugega in nato vlogi ulomkov zamenjamo, je razlika obeh produktov po absolutni vrednosti enaka $1$. To se bo izkazalo za pomembno opazko, zato vpeljemo pojem, ki sledi.

\begin{definicija}
Sosednja člena v Fareyevem zaporedju imenujemo \emph{Fareyeva soseda}.
\end{definicija}

% medianta
%
\begin{definicija}
Naj bosta $\frac{a}{b}$ in $\frac{c}{d}$ sosednja člena nekega Fareyevega zaporedja. Člen \[\frac{a+c}{b+d} \] imenujemo \emph{medianta}.
\end{definicija}

\begin{trditev}
Za medianto okrajšanih ulomkov \(\frac{a}{b} < \frac{c}{d}\) velja  \(\frac{a}{b} < \frac{a+c}{b+d} < \frac{c}{d}\) .
\end{trditev}

\begin{dokaz}
Poračunajmo razliki med členoma
\[\frac{a+c}{b+d} - \frac{a}{b} = \frac{ab+bc-ab-ad}{b(b+d)} = \frac{bc-ad}{b(b+d)} > 0\] in
\[\frac{c}{d} - \frac{a+c}{b+d} = \frac{bc+cd-ad-cd}{d(b+d)} = \frac{bc-ad}{d(b+d)} > 0.\]
Obe neenakosti sledita iz dejstva, da je \(\frac{a}{b} < \frac{c}{d}\), kjer so \(a, b, c, d \in \mathbb{N} \), zato je \( ad < bc.\)
Zveza torej velja. 
\end{dokaz}

% lastnosti 
%
Kako dobimo člen Fareyevega zaporedja reda $(n+1)$?
Označimo iskani okrajšan ulomek s $\frac{k}{n+1}$. Seveda sta $k, n \in\mathbb{N}, k < n+1$ tuji si števili. Zato obstajata enolično določeni naravni števili $a < b$, da velja $a(n+1)-bk=1.$ S preoblikovanjem zadnje enakosti dobimo zvezo $a(n+1-b)-b(k-a)=1,$ kar pomeni, da sta si tudi naravni števili $k-a$ in $n+1-b$ tuji. Brez škode za splošnost naj bo $k-a<n+1-b.$ Zato lahko tvorimo okrajšan ulomek $\frac{k-a}{n+1-b}$, ki pripada nekemu Fareyevemu zaporedju. Prav tako je okrajšan ulomek $\frac{a}{b}$ element nekega Fareyevega zaporedja. Sedaj prepišimo ulomek $\frac{k}{n+1}$ v $\frac{a+(k-a)}{b+(n+1-b)},$ kar pa je medianta ulomkov $\frac{a}{b}$ in $\frac{k-a}{n+1-b}.$ Dokazali smo naslednjo lema.

\begin{lema}
\label{lema:EltVišReda}
Dano naj bo Fareyevo zaporedje. Elemente zaporedja višjega reda dobimo z računanjem mediant elementov danega zaporedja.
\end{lema}

\begin{trditev}[Lastnost Fareyevih sosedov]
Naj velja \( 0 \leq \frac{a}{b} < \frac{c}{d} \leq 1\). Ulomka $\frac{a}{b}$ in $\frac{c}{d}$ sta Fareyeva soseda v nekem Fareyevem zaporeju natanko tedaj, ko velja \(bc - ad = 1\).
\end{trditev}

\begin{dokaz}
$(\Rightarrow)$ Dokaz bo potekal z indukcijo na $n$.
Za $n=1$ je $F_n = \{\frac{0}{1}, \frac{1}{1}\}$, $bc - ad = 1\cdot1 - 0\cdot1 = 1$, zato osnovni korak velja.
Po indukcijski predpostavki za zaporedje \( F_n = \{\ldots \frac{a}{b}, \frac{c}{d} \ldots \} \) velja $bc - ad = 1$. Dokažimo, da velja tudi za $F_{n+1}$. Vemo, da nove člene zaporedja dobimo z računanjem mediant. Če je $b+d > n+1,$ potem $\frac{a+c}{b+d} \notin F_{n+1}$ in je \( F_{n+1} = \{\ldots \frac{a}{b}, \frac{c}{d} \ldots \} \) ter po indukcijski predpostavki $bc - ad = 1$. 
Če je $b+d < n+1,$ je $\frac{a+c}{b+d}$ že nek člen v zaporedju $F_n$ in uporabimo indukcijsko predpostavko. 
Preostane še možnost $b+d = n+1$. Po lema~\ref{lema:EltVišReda} je edina možnost za člen med elementoma $\frac{a}{b}$ in $\frac{c}{d}$ njuna medianta $\frac{a+c}{b+d}$, ki pa je tudi edini nov člen v opazovanem delu zaporedja. To je zato oblike \( F_{n+1} = \{\ldots \frac{a}{b}, \frac{a+c}{b+d}, \frac{c}{d} \ldots \} \) in $b(a + c) - a(b + d) = ba + bc - ab - ad = bc - ad = 1$, kjer smo v zadnji enakosti uporabili indukcijsko predpostavko. Podobno je $(b + d)c - (a + c)d = bc + dc - ad - cd = bc - ad = 1$. Indukcijski korak je s tem končan. Torej sklep velja za vsa Fareyeva zaporedja.

$(\Leftarrow)$ Obratno, naj bodo $\frac{a}{b}$, $\frac{p}{q}$ in $\frac{c}{d}$ členi poljubnega Fareyevega zaporedja, za katere velja $\frac{a}{b} <\frac{p}{q} < \frac{c}{d}$ in $bp - aq = qc - pd = 1$. S preureditvijo te enakosti dobimo
\[ bp + pd = aq + qc, \]
\[ p(b + d) = q(a + c), \]
\[ \frac{p}{q} = \frac{a+c}{b+d}. \]
Vidimo, da je $\frac{p}{q}$ medianta ulomkov $\frac{a}{b}$ in $\frac{c}{d},$ od tod pa sledi, da sta $\frac{a}{b}$ in $\frac{p}{q}$ ter $\frac{p}{q}$ in $\frac{c}{d}$ Fareyeva soseda.
\end{dokaz}

\begin{lema}
\label{lema:MediantaOkrUlom}
Medianta je okrajšan ulomek.
\end{lema}

\begin{dokaz}
Naj za $\frac{a}{b} < \frac{c}{d}$ velja $bc - ad = 1$. Dokazati želimo, da je $\frac{a+c}{b+d}$ okrajšan ulomek, z drugimi besedami, da sta si števili $a+c$ in $b+d$ tuji. Če preoblikujemo zgornjo enakost, dobimo 
\[ 1 = bc - ad = ba + bc - ab - ad = b(a + c) - a(b + d), \]
kar pomeni, da $b+d$ in $a+c$ nimata skupnega faktorja. Ulomek $\frac{a+c}{b+d}$ je torej okrajšan.
\end{dokaz}

\begin{opomba}
Medianta $\frac{a+c}{b+d}$ je enolično določena z ulomkoma $\frac{a}{b}$ in $\frac{c}{d}$. To imenujemo \emph{lastnost mediante}.
\end{opomba}

%%%%%%%%%%%%%%%%%%%%%%%%%%%%%%%%%%%%%%%%%%%%%%%%%%%%%%%%%%%%%%%%%%%%%%%%%
% dolžina zaporedja
%
\subsection{Dolžina Fareyevega zaporedja}

Flitconova metoda za izračun števila okrajšanih ulomkov nas pripelje do naslednje rekurzivne formule dolžine Fareyevega zaporedja.

\begin{trditev}
\label{trd:DolzinaZap}
Naj bo $\varphi$ Eulerjeva funkcija. Dolžina Fareyevega zaporedja reda n je
\[  |F_{n}| = |F_{n-1}| + \varphi(n). \]
\end{trditev}

\begin{opomba}
\label{op:AsimptotDolzina}
Z upoštevanjem vrednosti $|F_{1}| = 2$ iz trditve~\ref{trd:DolzinaZap} sledi \[  |F_{n}| = \sum_{i=1}^n \varphi(i) + 1. \]
\end{opomba}

\begin{trditev}
\label{trd:AsimptotDolzina}
Asimptotično se dolžina Fareyevega zaporedja obnaša kot
\[  |F_{n}|\sim\frac{3n^2}{\pi^2}. \]
\end{trditev}

\begin{opomba}
Simbol $\sim$ v trditvi~\ref{trd:AsimptotDolzina} označuje asimptotično ekvivalentno obnašanje dveh funkcij.
Po definiciji za funkciji $f(x)$ in $g(x)$ velja $f(x) \sim g(x)$ natanko tedaj, ko je $ \lim_{x \to \infty} \frac{f(x)}{g(x)} = 1$.
\end{opomba}

% definicije pred dokazom
%
Preden dokažemo zgornjo trditev, definirajmo naslednjo oznako in dve funkciji, ki jih bomo v dokazu potrebovali.

\begin{definicija}
\emph{Notacija veliki O} predstavlja množico funkcij, ki so po absolutni vrednosti do multiplikativne konstante manjše od dane funkcije.

Natančneje, funkcija $f$ pripada razredu $O(g)$, če obstajata taki konstanti $M$ in $x_{0}$, da za vsak $x > x_{0}$ velja $|f(x)| \leq M \cdot |g(x)|$.

Pišemo $f \in O(g)$ oziroma $f = O(g)$.
\end{definicija}

\begin{definicija}
\label{def:MobFun}
Preslikava \( \mu\colon \mathbb{N} \to \mathbb{N} \), definirana kot
\[
\mu(n) = \left\{
\begin{array}{rl}
0 & ;\ \mbox{$n$ je deljiv s kvadratom praštevila}\\
(-1)^p & ;\  \mbox{$n$ je produkt $p$ različnih praštevil,}
\end{array}
\right.
\]
se imenuje \emph{M\"obiusova\footnote{August Ferdinand M\"obius, 17.\ 11.\ 1790 -- 26.\ 9.\ 1868, nemški matematik in astronom.} funkcija}.
\end{definicija}

\begin{primer}
Izračunajmo vrednosti M\"obiusove funkcije za nekaj naravnih števil.

\( \mu(1)=1, \)

\( \mu(2)=(-1)^{1}=-1=\mu(3), \)

\( \mu(4)=\mu(2^{2})=0, \)

\( \mu(6)=\mu(2\cdot3)=(-1)^{2}=1, \)

\( \mu(8)=\mu(2\cdot2^{2})=0, \)

\( \mu(18)=\mu(2\cdot3^{2})=0. \)
\end{primer}

\begin{definicija}
\label{def:RiemZeta}
\emph{Riemannova zeta funkcija} je za
 $s\in\mathbb{C}\backslash\{1\}$
definirana kot
\[ \zeta(s) = \sum_{n=1}^{\infty}\frac{1}{n^s}. \]
\end{definicija}

% dokaz asimptotskega obnašanja
%
Sedaj lahko dokažemo trditev~\ref{trd:AsimptotDolzina}. Dokaz sledi \cite[poglavje 18.5, str.~268]{hardy}.

\begin{dokaz}
Asimptotično obnašanje bomo izračunali s pomočjo ocene vrednosti vsote \( \sum_{i=1}^n \varphi(i) \).
Spomnimo se, da je 
\[ \varphi(n) = n \prod_{p|n} \left (1 - \frac{1}{p} \right ) = n - \sum \frac{n}{p} + \sum \frac{n}{pp'} - \cdots , \]
kjer so $p$, $p'$ praštevilski delitelji števila $n$.  Z upoštevanjem M\"obiusove funkcije je zadnja vsota enaka
\[ \varphi(n) = n \sum_{d|n} \frac{\mu(d)}{d} .\]
Sedaj računajmo vsoto 
%
\begin{align}
\label{Eq:DolZap}
\sum_{i=1}^n \varphi(i)
  &= \sum_{i=1}^n i \sum_{d|i} \frac{\mu(d)}{d} = \sum_{dd'\leq n}d' \mu(d) = 
    \sum_{d=1}^n \mu(d) \sum_{d'=1}^{\left \lfloor \frac{n}{d} \right \rfloor} d' \nonumber \\
  &= \frac{1}{2} \sum_{d=1}^{n} \mu(d) \left (\left \lfloor \frac{n}{d} \right \rfloor ^2 + \left \lfloor \frac{n}{d} \right \rfloor \right) =
    \frac{1}{2} \sum_{d=1}^{n} \mu(d) \left (\frac{n^2}{d^2} + O \left (\frac{n}{d} \right) \right) \nonumber \\
  &= \frac{1}{2}n^2 \sum_{d=1}^{n} \frac{\mu(d)}{d^2} + O \left (n \sum_{d=1}^{n} \frac{1}{d} \right ) \nonumber \\
  &\stackrel{(1)}{=} \frac{1}{2}n^2 \sum_{d=1}^{\infty} \frac{\mu(d)}{d^2} - \frac{1}{2}n^2 \sum_{d=n+1}^{\infty} \frac{\mu(d)}{d^2} + O(n \ln{n}) \nonumber \\
  &= \frac{1}{2}n^2 \sum_{d=1}^{\infty} \frac{\mu(d)}{d^2} + O \left (n^2\sum_{d=n+1}^{\infty} \frac{1}{d^2} \right ) + O(n \ln{n}) \nonumber \\
  & \stackrel{(2)}{=} \frac{n^2}{2 \zeta(2)} + O(n) + O(n \ln{n}) \stackrel{(3)}{=} \frac{3n^2}{\pi^2} + O(n \ln{n}).
\end{align}
%
V enakosti~(1) smo zadnji sumand ocenili navzgor s pomočjo Taylorjevega razvoja funkcije $\ln$ kot
\[ \ln{(1+x)} = \sum_{n=1}^{\infty} (-1)^{n+1} \frac{x^n}{n}.\]
%
V enakosti~(2) smo uporabili naslednjo oceno:
\[ n^2 \sum_{d=n+1}^{\infty} \frac{1}{d^2} \leq n^2 \sum_{d=n+1}^{\infty} \frac{1}{d(d-1)} = 
n^2 \sum_{d=n+1}^{\infty} \left (- \frac{1}{d} + \frac{1}{d-1} \right ) = n^2 \frac{1}{n} = n.\]
%
V enakosti~(3) smo za izračun funkcije $\zeta(2)$ uporabili znano vrednost
\[ \zeta(2) = \sum_{n=1}^{\infty} \frac{1}{n^2} = \frac{\pi^2}{6}.\]
%
Po opombi~\ref{op:AsimptotDolzina} iz izraza~\ref{Eq:DolZap} sledi, da je \(|F_n| = \frac{3n^2}{\pi^2} + O(n \ln{n}) \sim\frac{3n^2}{\pi^2}. \)
 %
\end{dokaz}

%%%%%%%%%%%%%%%%%%%%%%%%%%%%%%%%%%%%%%%%%%%%%%%%%%%%%%%%%%%%%%%%%%%%%%%%%%%%%%%%%%%%%%%%%%%%%%%%%%%%%%%%%%%%%%%%%%%%%%%%%%%%%%%%%%%%%%%%%%%%%%%%%%
%
\section{Fordovi krogi}

V tem poglavju bomo definirali Fordove kroge, ki so tesno povezani s Fareyevim zaporedjem. Večino lastnosti bomo dokazali z uporabo elementarnih geometrijskih sredstev in v analognih trditvah tistim iz prejšnjega poglavja prepoznali geometrijski pomen. Na kratko si bomo ogledali Fordove krogle -- Fordove kroge v treh dimenzijah. V zadnjem razdelku bomo z algebraičnim znanjem na drugačen način dokazali eno od lastnosti Fordovih krogov.
Ideje poglavja so povzete po \cite[poglavje 4]{fareyproject} in\cite{ford}.

\begin{definicija}
Naj bosta $p$ in $q$ tuji si števili v množici celih števil.
\emph{Fordov\footnote{Lester Randolph Ford~Sr., 25.\ 10.\ 1886 -- 11.\ 11.\ 1967, ameriški matematik.} krog} C($\frac{p}{q}$) je krog v zgornji polravnini, ki se abscisne osi dotika v točki $\frac{p}{q}$, njegov polmer pa meri $\frac{1}{2q^2}$. 
\end{definicija}

Ker so Fordovi krogi definirani za vsak okrajšan ulomek, lahko poljubnemu racionalnemu številu enolično priredimo Fordov krog. Iz analize vemo, da je množica racionalnih števil gosta podmnožica množice realnih števil, abscisna os pa je geometrijska predstavitev le-te. Zato poljubno majhen interval na abscisni osi vsebuje neskončno mnogo dotikališč Fordovih krogov.

Zaradi simetrije je Fordove kroge dovolj obravnavati na intervalu $[0,1]$, obenem pa se zavedati, da jih lahko induktivno razširimo na celotno realno os.

\begin{opomba}
Iz definicije zaradi pogoja o tujosti števil $p$ in $q$ neposredno sledi, da je množica Fordovih krogov v bijekciji s Fareyevim zapredjem.
\end{opomba}

% konstrukcija pravokotnega trikotnika
Oglejmo si konstrukcijo pravokotnega trikotnika, določenega s parom Fordovih krogov, ki bo ključna pri dokazovanju trditev v tem poglavju. Izberimo okrajšana ulomka $\frac{a}{b}$ in $\frac{c}{d}$, za katera velja $\frac{a}{b} < \frac{c}{d}$, ter jima priredimo ustrezna Fordova kroga. Naj bosta $A$ središče Fordovega kroga C($\frac{a}{b}$) in $B$ središče Fordovega kroga C($\frac{c}{d}$). Če je $b<d$, kar pomeni, da je polmer kroga C($\frac{a}{b}$) večji od polmera C($\frac{c}{d}$), točko $C$ določimo kot presečišče navpične premice skozi točko $A$ z vodoravno premico skozi točko $B$. Sicer je $b>d$, točka $C$ pa presečišče navpične premice skozi točko $B$ z vodoravno premico skozi točko $A$. Povežimo središči obeh krogov. Točki $D$ in $E$ naj bosta presečišči daljice $AB$ s Fordovima krogoma C($\frac{a}{b}$) in C($\frac{c}{d}$).

Vemo, da se kroga dotikata abscisne osi zaporedoma v točkah $\frac{a}{b}$ in $\frac{c}{d}$, njuna polmera pa merita $\frac{1}{2b^2}$ in $\frac{1}{2d^2}$. Od tod lahko izračunamo razdalje $|AB|$, $|AC|$ in $|BC|$.
Po konstrukciji je trikotnik $ABC$ pravokoten s pravim kotom v oglišču $C$, zato velja Pitagorov izrek
\begin{equation}
\label{Eq:Pitagora}
|AB|^2 = |AC|^2 + |BC|^2. 
\end{equation}

\begin{opomba}
V konstrukciji smo brez škode za splošnost predpostavili zvezo $\frac{a}{b} < \frac{c}{d}$. Če v splošnem primeru ta ne velja, zamenjamo vlogi ulomkov in postopamo na enak način kot v zgoraj opisanem primeru.
\end{opomba}
%

\begin{trditev}
\label{trd:FordDisjTang}
Fordova kroga, ki pripadata različnima okrajšanima ulomkoma, sta bodisi tangentna bodisi disjunktna.
\end{trditev}

\begin{dokaz}
Konstruirajmo pravokotni trikotnik, kot je opisano zgoraj in zapišimo Pitagorov izrek iz enačbe~\ref{Eq:Pitagora}.
Če dolžine izrazimo z $a, b, c$ in $d$, dobimo enakost
%
\begin{align}
|AB|^2 
  &= \left ( \left| \frac{1}{2b^2} - \frac{1}{2d^2} \right| \right)^2  + \left ( \left| \frac{c}{d} - \frac{a}{b} \right| \right )^2 \nonumber \\ 
  &= \frac{1}{4b^4} - \frac{1}{2b^2d^2} + \frac{1}{4d^4} + \left (\frac{bc-ad}{bd} \right )^2 \nonumber \\
  &= \left (\frac{1}{2b^2} + \frac{1}{2d^2} \right )^2 - \frac{1}{b^2d^2} + \frac{(bc-ad)^2}{b^2d^2} \nonumber \\
  &= (|AD| + |EB|)^2 + \frac{(bc-ad)^2-1}{b^2d^2}.
\end{align}

Če je $|bc-ad|>1$, je $|AB|^2 > (|AD| + |EB|)^2$, zato je $|AB| > |AD| + |EB|$ in Fordova kroga C($\frac{a}{b}$) in C($\frac{c}{d}$) sta disjunktna.

Če je $|bc-ad|=1$, je $|AB|^2 = (|AD| + |EB|)^2$, zato je $|AB| = |AD| + |EB|$ in Fordova kroga C($\frac{a}{b}$) in C($\frac{c}{d}$) sta tangentna.

Če je $|bc-ad|<1$, je $|bc-ad|=0,$ saj smo v množici celih števil. Sledi $\frac{a}{b} = \frac{c}{d},$ kar vodi v protislovje s predpostavko trditve.
\end{dokaz}

%%%%%%%%%%%%%%%%%%%%%%%%%%%%%%%%%%%%%%%%%%%%%%%%%%%%%%%%%%%%%%%%%%%%%%%%%
% Fordovi sosedi
\subsection{Fordovi sosedi}
%
Za tangentne Fordove kroge veljata naslednji lastnosti, ki lastnosti Fareyevih sosedov preneseta v jezik geometrije.

\begin{trditev}[Lastnost Fordovih sosedov]
\label{trd:FordTangentnost}
Fordova kroga C($\frac{a}{b}$) in C($\frac{c}{d}$) sta tangentna natanko tedaj, ko velja \( |bc-ad|=1. \)
\end{trditev}

\begin{dokaz}
%
Ponovno konstruirajmo pravokotni trikotnik kot v prejšnjem razdelku. Implikacijo v levo smo že izpeljali, zato si oglejmo še implikacijo v desno.

Denimo, da sta Fordova kroga C($\frac{a}{b}$) in C($\frac{c}{d}$) tangentna. Potem za pravokotni trikotnik, ki ga določata, velja Pitagorov izrek 
\[ |AC|^2 + |BC|^2 = |AB|^2 \] oziroma
\[ \left ( \left | \frac{1}{2b^2} - \frac{1}{2d^2} \right | \right )^2 + \left ( \left| \frac{c}{d} - \frac{a}{b} \right| \right )^2 = \left (\frac{1}{2b^2} + \frac{1}{2d^2} \right )^2. \]
Ko odpravimo oklepaje, opazimo, da se nekateri členi odštejejo. Nato odpravimo ulomke in dobljeno enakost poenostavimo.
\[ \frac{1}{4b^4} - \frac{1}{2b^{2}d^{2}} + \frac{1}{4d^4} + \frac{c^2}{d^2} - \frac{2ac}{bd} + \frac{a^2}{b^2} = \frac{1}{4b^4} + \frac{1}{2b^{2}d^{2}} + \frac{1}{4d^4}, \]
\[ b^{2}c^{2} - 2abcd + a^{2}d^{2} = 1, \]
\[  (bc-ad)^2 = 1, \]
\[ |bc-ad|=1. \]
Trditev je s tem dokazana.
\end{dokaz}

\begin{definicija}
Tangentna Fordova kroga imenujemo \emph{Fordova soseda}.
\end{definicija}

\begin{trditev}[Lastnost mediante za Fordove kroge]
Naj bosta C($\frac{a}{b}$) in C($\frac{c}{d}$) Fordova soseda. Tedaj obstaja enolično določen Fordov krog C($\frac{a+c}{b+d}$) in je tangenten na izbrana kroga. Imenujemo ga \emph{medianta Fordovih krogov}.
\end{trditev}

\begin{dokaz}
Po definiciji Fordovih krogov vemo, da sta $\frac{a}{b}$ in $\frac{c}{d}$ okrajšana ulomka in zaradi tangentnosti Fareyeva soseda v nekem Fareyevem zaporedju (razširjenem na celotno realno os). Po lema~\ref{lema:MediantaOkrUlom} je njuna medianta $\frac{a+c}{b+d}$ tudi okrajšan ulomek, torej obstaja natanko en Fordov krog C($\frac{a+c}{b+d}$). 

Dokažimo še, da je C($\frac{a+c}{b+d}$) tangenten na izbrana kroga. Ker sta C($\frac{a}{b}$) in C($\frac{c}{d}$) Fordova soseda, velja zveza 
\( |bc-ad|=1. \)
Če jo nekoliko preoblikujemo, dobimo
\[ |bc-ad|=|bc-ad+cd-cd|=|(b+d)c-(a+c)d|=1, \]
od koder sledi, da sta Fordova kroga C($\frac{a+c}{b+d}$) in C($\frac{c}{d}$) tangentna.
Podobno
\[ |bc-ad|=|bc-ad+ab-ab|=|(a+c)b-(b+d)a|=1 \]
pomeni, da sta Fordova kroga C($\frac{a}{b}$) in C($\frac{a+c}{b+d}$) tangentna.
\end{dokaz}

% konstrukcija Fordovih sosedov
Naslednji izrek pove, kako konstruiramo množico vseh Fordovih sosedov danega Fordovega kroga. Izrek in dokaz sledita \cite[trditev 3]{ford}.

\begin{izrek}
Naj bosta kroga C($\frac{p}{q}$) in C($\frac{P}{Q}$) Fordova soseda. Vse Fordove sosede Fordovega kroga C($\frac{p}{q}$) lahko zapišemo v obliki C($\frac{P_n}{Q_n}$), kjer je $\frac{P_n}{Q_n} = \frac{P+np}{Q+nq}$ in $n$ preteče vsa cela števila.
\end{izrek}

\begin{dokaz}
%
Najprej dokažimo, da sta Fordova kroga C($\frac{p}{q}$) in C($\frac{P_n}{Q_n}$) res Fordova soseda. Računajmo
\[ |qP_{n} - pQ_{n}| = |q(P+np) - p(Q+nq)| = |qP+qnp-pQ-pnq| = |qP-pQ| = 1. \]
Zadnja enakost velja po predpostavki, da sta C($\frac{p}{q}$) in C($\frac{P}{Q}$) Fordova soseda.

Sedaj preverimo, če obstajajo še Fordovi sosedi, ki niso zgornje oblike. Opazovali bomo zaporedje Fordovih krogov $\mathcal{M} = \{ \mathrm{C}(\frac{P_n}{Q_n}); n\in\Z \}.$
Iz računa
\begin{align}
|Q_{n}P_{n+1} - P_{n}Q_{n+1}| 
  &= |(Q+nq)(P+(n+1)p) - (P+np)(Q+(n+1)q)| \nonumber \\ 
  &= |QP + (n+1)Qp + nPq + n(n+1)pq \nonumber \\
  &   - PQ - (n+1)Pq - npQ - n(n+1)pq| \nonumber \\
  &= |Qp - pQ| \nonumber \\
  &= 1
\end{align}
sledi, da sta zaporedna elementa zaporedja $\mathcal{M}$ Fordova soseda. Ulomek $\frac{P_n}{Q_n}$, ki predstavlja Fordov krog C($\frac{P_n}{Q_n}$), lahko zapišemo kot
\begin{align}
\frac{P_n}{Q_n}
  &= \frac{P+np}{Q+nq} = \frac{Pq+npq}{q(Q+nq)} = \frac{Pq+npq+pQ-pQ}{q(Q+nq)} \nonumber \\
  &= \frac{p(Q+nq) + (Pq-pQ)}{q(Q+nq)} = \frac{p}{q} + \frac{Pq-pQ}{q(Q+nq)} \nonumber \\
  &= \frac{p}{q} \pm \frac{1}{q(Q+nq)} = \frac{p}{q} \pm \frac{1}{q^2 \left (n+\frac{Q}{q} \right)}.
\end{align}
%
V limiti, ko gre $n$ preko vseh meja, gre $\frac{P_n}{Q_n}$ proti $\frac{p}{q}.$
Ugotovili smo, da Fordovi krogi oblike C($\frac{P_n}{Q_n}$) geometrijsko tvorijo obroč okroli Fordovega kroga C($\frac{p}{q}$). Z njim so namreč vsi tangentni, prav tako pa so tangentni tudi na svojega predhodnika in naslednika v zaporedju $\mathcal{M}$. Njihova dotikališča z abscisno osjo konvergirajo proti točki $\frac{p}{q},$ ki je dotikališče danega Fordovega kroga C($\frac{p}{q}$), zaradi medsebojne tangentnosti pa so njihovi polmeri vse manjši. Zato ne obstaja Fordov krog, tangenten na C($\frac{p}{q}$), ki ni zgornje oblike in ne seka katerega izmed krogov iz zaporedja $\mathcal{M}$.
%
\end{dokaz}

% Pitagorejske trojice
%
V prejšnjem razdelku smo konstruirali pravokotni trikotnik, določen s središčema tangentnih Fordovih krogov in presečiščem premic skozi središči. Spomnimo se znane definicije iz teorije števil, ki izhaja iz evklidske geometrije. 

\begin{definicija}
Trojica naravnih števil $(a,b,c)$, za katero velja $a^2+b^2=c^2$, se imenuje \emph{pitagorejska trojica}\footnote{Pojem pitagorejska trojica nosi ime slavnega starogrškega matematika Pitagore (okoli 570~pr.~n.~št. -- 495~pr.~n.~št.), ki ga poznamo predvsem po Pitagorovem izreku.}.
Pitagorejska trojica je \emph{primitivna}, če števila $a$, $b$, in $c$ nimajo skupnega faktorja.
\end{definicija}

\begin{trditev}
Pravokotna trikotnika, ki pripadata poljubnima paroma Fordovih sosedov, določata različni primitivni pitagorejski trojici.
\end{trditev}

\begin{dokaz}
%
Naj bosta C($\frac{a}{b}$) in C($\frac{c}{d}$) ter C($\frac{a'}{b'}$) in C($\frac{c'}{d'}$) poljubna različna para Fordovih sosedov. Brez škode za splošnost naj velja $\frac{a}{b}<\frac{c}{d}$ in $\frac{a'}{b'}<\frac{c'}{d'}$. Naj prvemu paru Fordovih sosedov pripada pravokotni trikotnik $ABC$, drugemu paru pa pravokotni trikotnik $A'B'C'$. Dokazati želimo, da si trikotnika nista podobna.

Pa denimo, da sta si trikotnika $ABC$ in $A'B'C'$ podobna. Tedaj obstaja tako naravno število $\lambda\ne{1}$, da za dolžine stranic obeh pravokotnih trikotnikov veljajo naslednje zveze:
%
\begin{equation}
\label{Eq:Pit1}
\frac{1}{2b^2}-\frac{1}{2d^2} = \lambda \left (\frac{1}{2b'^2}-\frac{1}{2d'^2} \right ),
\end{equation}
%
\begin{equation}
\label{Eq:Pit2}
\frac{1}{2b^2}+\frac{1}{2d^2} = \lambda \left (\frac{1}{2b'^2}+\frac{1}{2d'^2} \right ),
\end{equation}
%
\begin{equation}
\label{Eq:Pit3}
\frac{c}{d}-\frac{a}{b} = \lambda \left (\frac{c'}{d'}-\frac{a'}{b'} \right ).
\end{equation}
%
Če seštejemo enačbi~\ref{Eq:Pit1} in \ref{Eq:Pit2}, dobimo
%
\begin{align}
\label{Eq:Pit4}
\frac{1}{b^2} &= \lambda \frac{1}{b'^2}, \nonumber \\
b'^2 &= \lambda b^2, \nonumber \\
b' &= \sqrt\lambda b.
\end{align}
%
Enačbo~\ref{Eq:Pit3} lahko poenostavimo, saj gre za para Fordovih sosedov. Velja
\begin{equation}
\label{Eq:Pit5}
\frac{1}{bd} = \frac{bc-ad}{bd} = \frac{c}{d}-\frac{a}{b} = \lambda \left (\frac{c'}{d'}-\frac{a'}{b'} \right ) = \lambda \frac{b'c'-a'd'}{b'd'} = \lambda \frac{1}{b'd'}.
\end{equation}
%
Iz enakosti~\ref{Eq:Pit4} in~\ref{Eq:Pit5} sledi 
%
\begin{align}
\label{Eq:Pit6}
\frac{1}{bd} &=\lambda \frac{1}{\sqrt\lambda bd'}, \nonumber \\
\frac{1}{d} &= \frac{\sqrt\lambda}{d'}, \nonumber \\
d' &= \sqrt\lambda d.
\end{align}
%
Nazadnje še v pogoj za tangentnost Fordovih krogov C($\frac{a'}{b'}$) in C($\frac{c'}{d'}$) vstavimo zvezi~\ref{Eq:Pit4} in~\ref{Eq:Pit6}, kar nam da
%
\begin{align}
b'c'-a'd' &= 1, \nonumber \\
\sqrt\lambda bc' - a' \sqrt\lambda d &= 1, \nonumber \\
\sqrt\lambda (bc'-a'd) &= 1.
\end{align}
%
To pa je možno le tedaj, ko je $\lambda=1.$ Prispeli smo do protislovja, kar pomeni, da si trikotnika nista podobna.
%
Zakaj so pitagorejske trojice primitivne?
Če so dolžine stranic posameznega pravokotnega trikotnika paroma tuja si cela števila, že določajo primitivno pitagorejsko trojico.
Če imajo ta cela števila skupni celoštevilski faktor, jih z njim delimo (to geometrijsko pomeni, da konstruiramo podobni trikotnik), kar nam da primitivno pitagorejsko trojico.
Če pa so dolžine stranic racionalna števila, jih pomnožimo z najmanjšim skupnim večkratnikom njihovih imenovalcev in dobimo enega izmed zgornjih primerov.
\end{dokaz}

%%%%%%%%%%%%%%%%%%%%%%%%%%%%%%%%%%%%%%%%%%%%%%%%%%%%%%%%%%%%%%%%%%%%%%%%%
% Posplositve
\subsection{Posplošeni Fordovi krogi}

V prejšnjem razdelku smo se ukvarjali s Fordovimi krogi, ki so bili enolično določeni z racionalnim številom. Natančneje, za dan okrajšan ulomek $\frac{p}{q}$ smo konstruirali Fordov krog na zgornji polravnini evklidske ravnine, ki se abscisne osi dotika v točki $\frac{p}{q}$, njegov polmer pa meri $\frac{1}{2q^2}.$
Nadaljujemo lahko s splošnejšimi Fordovimi krogi, ki so definirani na povsem enak način, le da imajo polmer enak $\frac{1}{2hq^2},$ pri čemer je $h$ poljubno realno število. Imenujemo jih tudi Speiserjevi\footnote{Andreas Speiser, 10.\ 6.\ 1885 -- 12.\ 10.\ 1970, švicarski matematik, ki se je ukvarjal s teorijo števil in teorijo grup.} krogi. Če izberemo $h=1,$ dobimo običajne Fordove kroge.

%%%%%%%%%%%%%%%%%%%%%%%%%%%%%%%%%%%%%%%%%%%%%%%%%%%%%%%%%%%%%%%%%%%%%%%%%
% Fordove krogle
\subsection{Fordove krogle}

V tem razdelku bomo sledili \cite[poglavje 8]{ford}.
Tokrat naj bosta števili $p$ in $q$ elementa množice Gaussovih celih števil, ki je definirana kot $\mathbb{Z}\left[{i}\right] = \{a+bi; a,b \in \Z \}.$ Zapišimo $p = p'+ip''$ in $q = q'+iq'',$ kjer so $p', p'', q', q'' \in \Z.$ Definirajmo ulomek
\begin{align} 
\frac{p}{q} = \frac{p'+ip''}{q'+iq''} = \frac{(p'+ip'')(q'-iq'')}{(q'+iq'')(q'-iq'')} = \frac{p'q'+p''q''}{q'^2+q''^2} + i \frac{p''q'-p'q''}{q'^2+q''^2},
\end{align}
ki pripada kompleksnim številom, in ga okrajšajmo. Geometrijsko predstavlja točko v Gaussovi $xy$-ravnini z realno in imaginarno komponento. Gaussovo ravnino postavimo v prostor, določen z osjo $z$, pravokotno na Gaussovo ravnino, in opazujmo podprostor, ki pripada pozitivnim vrednostim na $z$-osi. Analogno Fordovim krogom v ravnini pridemo do naslednjega pojma. 

\begin{definicija}
\emph{Fordova krogla} S($\frac{p}{q}$), kjer je $\frac{p}{q}$ okrajšan ulomek v množici kompleksnih števil, je krogla v zgornjem polprostoru, definiranem kot zgoraj, ki se $xy$-ravnine dotika v točki, določeni s $\frac{p}{q}$, njen polmer pa meri $\frac{1}{2|q|^2}.$
\end{definicija}

% lastnosti
Na analogen način trditvam, ki opisujejo lastnosti Fordovih krogov, lahko izpeljemo in dokažemo lastnosti Fordovih krogel. Omenimo le nekatere izmed njih.

Poljubno majhen zaprt pravokotnik v $xy$-ravnini vsebuje neskončno mnogo dotikališč Fordovih krogel.

Spomnimo se konstrukcije pravokotnega trikotnika, določenega z dvema Fordovima krogoma. Naj točki $\frac{p}{q}$ in $\frac{P}{Q}$ določata Fordovi krogli ter konstruirajmo pravokotni trikotnik $ABC$ kot prej. Iz zveze
\begin{align*}
|AB|^2 
  &= \left |\frac{P}{Q} - \frac{p}{q} \right|^2 + \left |\frac{1}{2|Q|^2} - \frac{1}{2|q|^2} \right|^2 = \frac{|Pq-pQ|^2-1}{|Q|^2|q|^2} + (|AD| + |EB|)^2
\end{align*}
sledi:
če je $|Pq-pQ|>1,$ je $|AB|>|AD|+|EB|$ in krogli sta disjunktni;
sicer je $|Pq-pQ|=1,$ zato je $|AB|=|AD|+|EB|$ in krogli sta tangentni.

Naj bosta S($\frac{p}{q}$) in S($\frac{P}{Q}$) tangentni Fordovi krogli. Kot prej tangentne Fordove krogle na kroglo S($\frac{p}{q}$) konstruiramo s pomočjo formule
\begin{equation}
\frac{P_n}{Q_n} = \frac{P+np}{Q+nq},
\end{equation}
le da tokrat $n$ pripada množici Gaussovih celih števil. 
Nadalje nas zanima, koliko Fordovih krogel je tangentnih na kroglo S($\frac{P_n}{Q_n}$).
Uporabimo pogoj za tangentnost, ki smo ga izpeljali. Računajmo
\begin{align} 
|P_{n}Q_{m} - P_{m}Q_{n}| 
  &= |(P+np)(Q+mq) - (P+mp)(Q+nq)| \nonumber \\
  &= |PQ+mPq+npQ+mnpq-PQ-nqP-mpQ-mnpq| \nonumber \\
  &= |Pq-pQ| |m-n| = 1.
\end{align}
%
Sledi, da je $|m-n|=1,$ torej je razlika $m-n \in \{1, -1, i, -i \}.$ Ugotovili smo, da je vsaka Fordova krogla, ki je tangentna na dano Fordovo kroglo, tangentna še na štiri druge Fordove krogle. Velja, da so te edine tangentne krogle.

%%%%%%%%%%%%%%%%%%%%%%%%%%%%%%%%%%%%%%%%%%%%%%%%%%%%%%%%%%%%%%%%%%%%%%%%%% Mobiusove transformacije na Fordovih krogih
\subsection{M\"{o}biusove transformacije na Fordovih krogih}

Do sedaj smo Fordove kroge obravnavali s pomočjo geometrijskih sredstev. V tem poglavju si bomo z algebro pomagali do nekaterih že znanih rezultatov o Fordovih krogih.
Geometrijske objekte si bomo predstavljali v kompleksni ravnini, torej bo točka s koordinatama $(x,y)$ opisana s kompleksnim številom $z=x+iy$.

% Mobiusova transformacija
Najprej se spomnimo naslednjega pojma iz kompleksne analize.
\begin{definicija}
\label{def:MobTransformacija}
Preslikava \( f \colon \mathbb{CP}^{1} \rightarrow \mathbb{CP}^{1} \), definirana s predpisom \( f(z) = \frac{az+c}{bz+d} \), kjer so $a,b,c,d \in \mathbb{C}$ in $ad-bc \neq 0$, se imenuje \emph{M\"{o}biusova transformacija}.
\end{definicija} 

\begin{opomba}
Simbol $ \mathbb{CP}^{1}$ označuje \emph{Riemannovo sfero}, to je kompaktifikacijo kompleksne ravnine z eno točko, kar zapišemo kot $\mathbb{CP}^{1} = \mathbb{C} \cup \{\infty\}$.
\end{opomba}

\begin{opomba}
Števila $a,b,c,d$ lahko pomnožimo s poljubnim neničelnim kompleksnim številom, zato lahko predpostavimo, da je $ad-bc=1$.

V našem primeru bo dovolj obravnavati le $a,b,c,d \in \mathbb{Z}$.
\end{opomba}

M\"{o}biusova transformacija je meromorfna in bijektivna preslikava z inverzom, ki je spet take oblike, identična preslikava $id(z)=z$ je M\"{o}biusova transformacija, prav tako je kompozitum M\"{o}biusovih transformacij M\"{o}biusova transformacija. Množica takih preslikav tako tvori grupo za kompozitum. 
Preslikavo $f$ lahko zapišemo v matrični obliki
\[
\mathbf{A}\ =
\left[
\begin{array}{cc}
a & c \\
b & d \\
\end{array}
\right],
\]
kjer so $a,b,c,d \in \mathbb{Z}$. Ker velja $ad-bc=1$, je matrika $A \in {SL}_{2}(\mathbb{Z})$. Zato bomo M\"{o}biusove transformacije predstavljali s splošno linearno grupo ${SL}_{2}(\mathbb{Z})$.

% Delovanje grupe na množico
Ključen pri izpeljavi rezultatov o Fordovih krogih je pojem, ki ga v algebri pogosto uporabljamo.

\begin{definicija}
\emph{Delovanje grupe} $G$ na množico $M$ je taka preslikava \( \circ \colon G \times M \rightarrow M \), za katero velja:
\begin{enumerate}
	\item \( e \circ \alpha = \alpha \) za vsak $ \alpha \in M$, kjer je $e$ enota grupe $G$, 
	\item \( g \circ (h \circ \alpha) = (gh) \circ \alpha \) za vsak $\alpha \in M$ in vsaka $g,h \in G$.
\end{enumerate}

Ekvivalentno, delovanje grupe $G$ na množico $M$ je homomorfizem iz grupe $G$ v grupo permutacij množice $M$.
\end{definicija}

\begin{primer}
Naj bo $G$ grupa permutacij $n$ elementov, torej $G = S_{n}$, $M$ pa naj bo množica $M = \{1,2, \ldots, n \}$.
Delovanje grupe $G$ na množico $M$ je preslikava \( \circ \colon G \times M \rightarrow M \) s predpisom \( (\pi, \alpha) \mapsto \pi(\alpha) = \pi \circ \alpha \).
\end{primer}

% Fordovi krogi
\begin{definicija}
Fordov krog C($\frac{1}{0}$), katerega polmer je neskončen, je premica $\mathbb{R} + i$.
\end{definicija}

\begin{izrek}
\label{izr:MobDelovanje}
M\"{o}biusova transformacija slika Fordove kroge v Fordove kroge.
\end{izrek}

\begin{dokaz}
Naj bo M\"{o}biusova transformacija dana z matriko 
\[
\mathbf{A}\ =
\left[
\begin{array}{cc}
a & c \\
b & d \\
\end{array}
\right]
\in {SL}_{2}(\mathbb{Z}).
\]
Vemo, da tovrstna preslikava slika premice in krožnice v premice in krožnice. Ideja dokaza je pokazati, da grupa ${SL}_{2}(\mathbb{Z})$ deluje na množico Fordovih krogov.

S krajšim računom se lahko prepričamo, da je grupa ${SL}_{2}(\mathbb{Z})$ generirana z matrikama 
\(
\left[
\begin{array}{cc}
1 & 1 \\
0 & 1 \\
\end{array}
\right]
\)
in
\(
\left[
\begin{array}{cc}
0 & -1 \\
1 & 0 \\
\end{array}
\right]
\).
To pomeni, da je vsako M\"{o}biusovo transformacijo moč zapisati kot kompozitum preslikav $z \mapsto z+1$ in $z \mapsto -\frac{1}{z}$, ki ustrezata generatorjema. Trdimo, da M\"{o}biusova transformacija preslika Fordov krog z intervala $ [0,1]$ v Fordov krog z intervala $[n,n+1]$, kjer $n$ pripada množici celih števil. 

Preslikava $z \mapsto z+1$ je translacija, zato zgornje očitno velja.

Kaj pa preslikava $z \mapsto -\frac{1}{z}$? Najprej si oglejmo splošen primer. Naj bo original Fordov krog s polmerom $r$ in središčem v $\alpha$, predstavljen z enačbo $ |z-\alpha|=r$. Prepišimo enačbo v
\begin{align}
(z-\alpha)( \bar{z}-\bar{\alpha}) &= r^2, \nonumber \\
z\bar{z} - \alpha \bar{z} - \bar{\alpha} z + \alpha \bar{\alpha} - r^2 &= 0
\end{align}
in označimo $R = \alpha \bar{\alpha} - r^2$. Polmer lahko izrazimo kot $r = \sqrt{|\alpha|^2-R}$.
Preslikajmo original s preslikavo $z \mapsto -\frac{1}{z}$. Enačba slike se glasi
\begin{align}
\left(-\frac{1}{z}\right)\left(-\frac{1}{\bar{z}}\right) + \alpha \frac{1}{\bar{z}} + \bar{\alpha} \frac{1}{z} + R &= 0, \nonumber \\
1 + \alpha z + \bar{\alpha} \bar{z} + Rz \bar{z} &= 0, \nonumber \\
z \bar{z} + \frac{\bar{\alpha}}{R} \bar{z} + \frac{\alpha}{R} z + \frac{1}{R} &= 0.
\end{align}
Sedaj vzemimo Fordov krog C($\frac{p}{q}$). Njegov polmer meri $\frac{1}{2q^2}$, središče pa je v točki $\frac{p}{q} +i \frac{1}{2q^2}$.
Izračunajmo
\[ R = \alpha \bar{\alpha} - r^2 = \left(\frac{p}{q} + i\frac{1}{2q^2} \right) \left(\frac{p}{q} - i\frac{1}{2q^2} \right) - \frac{1}{4q^4} 
= \frac{p^2}{q^2} + \frac{1}{4q^4} - \frac{1}{4q^4} = \frac{p^2}{q^2} \]
in preverimo, da je slika izbranega Fordovega kroga tudi Fordov krog. Res, središče slike je po enačbi (9) v točki
\[ -\frac{\bar{\alpha}}{R} = - \frac{\frac{p}{q} - i \frac{1}{2q^2}}{\frac{p^2}{q^2}} = - \frac{q}{p} + i \frac{1}{2p^2}, \]
polmer pa meri
\[ \sqrt{ \left| - \frac{\bar{\alpha}}{R} \right|^2 - \frac{1}{R} } = \sqrt{ \frac{\bar{\alpha}}{R} \frac{\alpha}{R} - \frac{1}{R} } 
= \frac{1}{R} \sqrt{|\alpha|^2 - R} =  \frac{1}{R} \frac{1}{2q^2} = \frac{q^2}{2p^2 q^2} = \frac{1}{2p^2}. \]

Dokazali smo, da preslikavi, ki generirata grupo M\"{o}biusovih transformacij, Fordov krog preslikata v Fordov krog z intervala $[n,n+1]$ za $n \in \mathbb{Z}$. Da dobimo Fordov krog z intervala $[0,1]$, moramo prvotno preslikavo komponirati z $n$ translacijami v levo (za $-1$). Ker identiteta slika Fordov krog vase in je kompozitum M\"{o}biusovih transformacij dobro definiran, grupa ${SL}_{2}(\mathbb{Z})$ res deluje na množico Fordovih krogov.
\end{dokaz}

% lastnost mediante in Mobiusova transformacija
\begin{lema}
Dana naj bosta Fordova kroga C($\frac{e}{f}$) in C($\frac{g}{h}$). M\"{o}biusova transformacija, ki deluje na množico Fordovih krogov, ohranja $|eh-fg|$.
\end{lema}

\begin{dokaz}
Kot v Definiciji~\ref{def:MobTransformacija} zapišimo M\"{o}biusovo transformacijo v obliki $f(z) = \frac{az+c}{bz+d}$, kjer so $a,b,c,d \in \mathbb{Z}$ in velja $ad-bc=1$. 
Vzemimo Fordova kroga C($\frac{e}{f}$) in C($\frac{g}{h}$) in poglejmo, kam se preslikata. 
Ker velja Izrek~\ref{izr:MobDelovanje} in je Fordov krog enolično določen s točko, v kateri se dotika realne osi v kompleksni ravnini, je dovolj poznati sliko te točke.
Izračunajmo
\[ f \left(\frac{e}{f} \right) = \frac{a \frac{e}{f} + c}{b \frac{e}{f} + d} = \frac{ae + cf}{be + df}, \]
od koder sledi, da se Fordov krog C($\frac{e}{f}$) preslika v Fordov krog C($\frac{ae+cf}{be+df}$).
Podobno,
\[ f \left(\frac{g}{h} \right) = \frac{a \frac{g}{h} + c}{b \frac{g}{h} + d} = \frac{ag + ch}{bg + dh}, \]
in C($\frac{g}{h}$) se preslika v C($\frac{ag+ch}{bg+dh}$).
Označimo \( e' = ae+cf, f' = be+df, g' = ag+ch \) in $h' = bg+dh$. Računajmo
\[ e'h'-f'g' = (ae+cf)(bg+dh) - (be+df)(ag+ch) = (ad - bc)(eh - fg), \]
katerega absolutna vrednost je enaka
\[ |e'h'-f'g'| = |ad-bc| |eh-fg| = |eh-fg|, \]
s čimer smo dokazali želeno enakost.
\end{dokaz}

Iz zgornjega neposredno sledi naslednja ugotovitev.

\begin{posledica}
\label{Posl:MobOhr1}
Dana naj bosta Fordova kroga C($\frac{e}{f}$) in C($\frac{g}{h}$). M\"{o}biusova transformacija, ki deluje na množico Fordovih krogov, ohranja $|eh-fg|=1$.
\end{posledica}

Sedaj bomo s pomočjo orodij, ki smo jih obravnavali v tem razdelku, ponovno dokazali že znano \emph{lastnost mediante za Fordove kroge}.

\begin{dokaz}[Dokaz lastnosti mediante za Fordove kroge z M\"{o}biusovimi transformacijami]
Kroga C($\frac{r}{s}$) in C($\frac{p}{q}$) naj bosta Fordova soseda. Trdimo, da je kandidat C($\frac{r+p}{s+q}$) medianta izbranih Fordovih krogov.

Vzemimo točke v presečiščih teh treh krogov. Vemo, da za točke $\alpha$, $\beta$, $\gamma$ obstaja M\"{o}biusova transformacija, ki jih preslika v poljubne točke $\lambda$, $\mu$, $\nu$. Torej obstaja M\"{o}biusova transformacija $M_{1}$, ki izbrane točke v presečiščih preslika v točke $i$, $\frac{1}{2}+\frac{1}{2}i$, $1+i$. Ker M\"{o}biusova transformacija slika Fordove kroge v Fordove kroge, mi pa poznamo sliki dveh točk na vsakem izmed krogov C($\frac{r}{s}$), C($\frac{p}{q}$) in C($\frac{r+p}{s+q}$), poznamo slike vseh treh Fordovih krogov. Te pa so točno C($\frac{1}{0}$), C($\frac{0}{1}$) in C($\frac{1}{1}$).

Sedaj bomo konstruirali zaporedje M\"{o}biusovih transformacij (ki je prav tako M\"{o}biusova transformacija). Začnimo s Fordovima sosedoma C($\frac{r}{s}$) in C($\frac{p}{q}$). Preslikava $M_{1}$ ju preslika v Fordova kroga C($\frac{1}{0}$) in C($\frac{0}{1}$) z medianto, Fordovim krogom, C($\frac{1}{1}$). Kot v prvem delu dokaza izberimo tri točke v presečiščih in jih preslikajmo. Obstaja M\"{o}biusova transformacija $M_{2}$, ki Fordove kroge C($\frac{1}{0}$), C($\frac{0}{1}$) in C($\frac{1}{1}$) zaporedoma preslika v Fordove kroge C($\frac{a}{b}$) in C($\frac{c}{d}$) in C($\frac{a+c}{b+d}$). Nadaljujemo s transformacijo $M_{3}$, ki slednje preslika po vrsti v Fordove kroge C($\frac{1}{0}$), C($\frac{0}{1}$) in C($\frac{1}{1}$), in končajmo s transformacijo $M_{4}$, ki le-te preslika v Fordove kroge C($\frac{r}{s}$), C($\frac{p}{q}$) in C($\frac{r+p}{s+q}$).

Vse zgornje preslikave so M\"{o}biusove transformacije, ki ohranjajo medianto Fordovih krogov (saj velja Posledica~\ref{Posl:MobOhr1} ter je C($\frac{1}{1}$) medianta C($\frac{1}{0}$) in C($\frac{0}{1}$)), zato tudi kompozitum $M_{4} \circ M_{3} \circ M_{2} \circ M_{1}$ ohranja medianto. Sledi, da je naš kandidat C($\frac{r+p}{s+q}$) res medianta. Z drugimi besedami, za poljubna Fordova soseda obstaja enolično določen Fordov krog, ki je tangenten na oba.
\end{dokaz}

%%%%%%%%%%%%%%%%%%%%%%%%%%%%%%%%%%%%%%%%%%%%%%%%%%%%%%%%%%%%%%%%%%%%%%%%%%%%%%%%%%%%%%%%%%%%%%%%%%%%%%%%%%%%%%%%%%%%%%%%%%%%%%%%%%%%%%%%%%%%%%%%%%%%%%
%
\section{Riemannova hipoteza}

Riemannova hipoteza, znana tudi kot 8.~Hilberov problem, je eno najbolj slavnih, še vedno nerešenih matematičnih vprašanj. Ime je dobila po nemškem matematiku Bernhardu Riemannu, ki jo je formuliral med preučavenjem lastnosti velikih praštevil.
%
%%%%%%%%%%%%%%%%%%%%%%%%%%%%%%%%%%%%%%%%%%%%%%%%%%%%%%%%%%%%%%%%%%%%%%%%%%%% Praštevila in Riemannova zeta funkcija
\subsection{Praštevila in Riemannova zeta funkcija}

% praštevila
Praštevila so poznali že v Stari Grčiji, od koder prihaja tudi naslednji izrek.
\begin{izrek}[Evklid]
Praštevil je neskončno mnogo.
\end{izrek}

Dokazov tega fundamentalnega izreka je veliko, velja pa omeniti Evklidovo idejo dokazovanja s protislovjem, ki jo pogosto uporabljamo še danes.

% zeta funkcija
V 1.~polovici 18.~stoletja je Euler definiral realno funkcijo zeta s predpisom
\begin{align}
\zeta \colon \mathbb{R} \rightarrow \mathbb{R}, \nonumber \\
\zeta(n) = \sum_{r=1}^{\infty}\frac{1}{r^n}.
\end{align}
Za $n=1$ je enaka harmonični vrsti, ki divergira; za $n>1$ pa dobimo konvergentno vrsto. Funkcija zeta je povezana s praštevili preko naslednje formule, ki jo je Euler objavil leta 1737 v knjigi \emph{Variae observationes circa series infinitas}.

\begin{izrek}[Eulerjeva produktna formula]
\label{izr:EulProdukt}
Naj bo $n\in\mathbb{N}$ in $p\in\mathbb{P}$. Tedaj velja
\begin{equation}
\sum_{n}\frac{1}{n^s} = \prod_{p}\frac{1}{1-p^{-s}}.
\end{equation}
\end{izrek}

\begin{dokaz}
Zapišimo predpis za funkcijo zeta. V naslednjem koraku funkcijo zeta pomnožimo z $\frac{1}{2^s}$ ter dobljeno enakost odštejemo od prve enakosti. S tem odpadejo vsi členi s faktorjem $2$. Ta dva koraka ponavljamo: enakost na trenutnem koraku pomnožimo z drugim sumandom na desni strani enakosti, nato pa predzadnji vrstici odštejemo zadnjo vrstico.
\begin{align*}
	\zeta(s) &= 1 + \frac{1}{2^s} + \frac{1}{3^s} + \frac{1}{4^s} + \frac{1}{5^s} + \ldots, \nonumber \\
	\frac{1}{2^s} \zeta(s) &= \frac{1}{2^s} + \frac{1}{4^s} + \frac{1}{6^s} + \frac{1}{8^s} + \frac{1}{10^s} + \ldots, \nonumber \\
	\left(1-\frac{1}{2^s}\right) \zeta(s) &= 1 + \frac{1}{3^s} + \frac{1}{5^s} + \frac{1}{7^s} + \frac{1}{9^s} + \ldots, \nonumber \\
	\frac{1}{3^s} \left(1-\frac{1}{2^s}\right) \zeta(s) &= \frac{1}{3^s} + \frac{1}{9^s} + \frac{1}{15^s} + \frac{1}{21^s} + \frac{1}{27^s} \ldots, \nonumber \\
	\left(1-\frac{1}{3^s}\right)\left (1-\frac{1}{2^s}\right) \zeta(s) &= 1 + \frac{1}{5^s} + \frac{1}{7^s} + \frac{1}{11^s} + \frac{1}{13^s} + \ldots \nonumber \\
\end{align*}
Opazimo, da smo vselej množili z ulomki oblike $\frac{1}{p^s},$ kjer $p$ pripada množici praštevil. Dobimo enakost
	\[ \ldots \left(1-\frac{1}{13^s}\right) \left(1-\frac{1}{11^s}\right) \left(1-\frac{1}{7^s}\right) \left(1-\frac{1}{5^s}\right) \left(1-\frac{1}{3^s}\right) 		\left(1-\frac{1}{2^s}\right) \zeta(s) = 1, \]
od koder lahko izrazimo
	\[ \zeta(s) = \frac{1}{1-\frac{1}{2^s}} \frac{1}{1-\frac{1}{3^s}} \frac{1}{1-\frac{1}{5^s}} \frac{1}{1-\frac{1}{7^s}} \frac{1}{1-\frac{1}{11^s}} \ldots = 		\prod_{p}\frac{1}{1-p^{-s}}, \]
kar smo želeli pokazati.
\end{dokaz}

% Riemannova zeta funkcija
V 19.~stoletju so začeli računati s kompleksnimi števili, tako je Riemann leta 1859 razširil Eulerjevo definicijo funkcije zeta. Riemannova funkcija zeta množico  $\mathbb{C}\backslash\{1\}$ preslika na množico  $\mathbb{C}$ in je definirana v Definiciji~\ref{def:RiemZeta}.

\begin{opomba}
V Izreku~\ref{izr:EulProdukt} nismo povedali, kateri množici pripada spremenljivka $s$. Euler je formulo namreč formuliral za celoštevilske $s$, Riemann pa je z razširitvijo funkcije zeta dokazal, da enakost velja za vse $s$, za katere velja $Re(s)>1$.
\end{opomba}

Zanimajo nas ničle Riemannove funkcije zeta. Na polravnini $Re(s)>1$ se funkcija ujema s $\prod_{p}\frac{1}{1-p^{-s}}$. Ker so vsi faktorji $\frac{1}{1-p^{-s}} \neq 0$ in je $\lim_{p\to\infty} \frac{1}{1-p^{-s}} = 1$, je na tej polravnini $\prod_{p}\frac{1}{1-p^{-s}} \neq 0$. To pomeni, da Riemannova funkcija zeta nima ničel za $Re(s)>1$.

Kako je z ničlami na polravnini $Re(s)<0$?
Pomagali si bomo z Bernoullijevimi števili. Naj bo $|z|<2\pi$. \emph{Bernoullijeva\footnote{Jacob~Bernoulli, 27.~12.~1654 -- 16.~8.~1705, rojen v družini znamenitih švicarskih matematikov. Med drugim mu pripisujemo odkritje konstante $e$.} števila} $B_{k}$ definiramo preko funkcije
\begin{equation}
G(z) = \frac{z}{e^z-1} = \sum_{k=0}^{\infty} B_{k} \frac{z^k}{k!}.
\end{equation}
Z razvojem v Taylorjevo vrsto dobimo
\[ G(z) = \left(1 + \frac{z}{2!} + \frac{z^2}{3!} + \frac{z^3}{4!} + \ldots \right)^{-1} = 1 - \frac{z}{2} + \frac{z^2}{12} - \frac{z^4}{720} + \ldots, \]
od koder preberemo prvih nekaj Bernoullijevih števil: $B_{0} = 1$, $B_{1} = -\frac{1}{2}$,  $B_{2} = \frac{1}{6}$,  $B_{3} = 0$, $B_{4} = -\frac{1}{30}$.
Za vse lihe $n$, kjer je $n \geq 3$, velja $B_{n}=0$. 
Za nas je pomembna naslednja povezava Bernoullijevih števil z Riemannovo funkcijo zeta. Naj bo $n\in\{0, 1, 2, \dots\}$. Velja
\begin{equation}
	\zeta(-n) = (-1)^n \frac{B_{n+1}}{n+1}.
\end{equation}
Ker so za sode $n$ vrednosti $B_{n+1}=0$, je $\zeta(s)=0$ za $s\in\{-2,-4,-6,\dots\}$. To pa so tudi edine ničle na opazovani polravnini. Imenujemo jih \emph{trivialne ničle} Riemannove funkcije zeta. Ostal je še pas $0<Re(s)<1$, o ničlah na njem pa govori naslednji izrek. 

\begin{izrek}[Riemannova hipoteza]
Vse netrivialne ničle Riemannove funkcije zeta ležijo na premici $s=\frac{1}{2}+it$.
\end{izrek}

%%%%%%%%%%%%%%%%%%%%%%%%%%%%%%%%%%%%%%%%%%%%%%%%%%%%%%%%%%%%%%%%%%%%%%%%%%%% Riemannova hipoteza
\subsection{Riemannova hipoteza}

Obstaja več ekvivalentnih formulacij Riemannove hipoteze. Dokazali bomo eno izmed njih, ki vključuje Fareyevo zaporedje. Pred tem se spomnimo M\"obiusove funkcije, definirane v Definiciji~\ref{def:MobFun}.

\begin{definicija}
Za $n\in\mathbb{N}$ je \emph{Mertensova\footnote{Franz~Mertens, 20.~3.~1840 -- 5.~3.~1927, poljski matematik.} funkcija} definirana s predpisom
\[ M(n)=\sum_{k\leq n}\mu(k).\]
\end{definicija}

\begin{primer}
Izračunajmo nekaj vrednosti Mertensove funkcije.

\( M(2) = \mu(1) + \mu(2) = 1 - 1 = 0 \)

\( M(3) = \mu(1) + \mu(2) + \mu(3) = 1 - 1 - 1 = -1 \)

\( M(6) = \mu(1) + \mu(2) + \mu(3) + \mu(4) + \mu(5) + \mu(6) = 1 - 1 - 1 + 0 - 1 + 1 = -1 \)
\end{primer}

Naslednjo trditev, ki jo bomo navedli brez dokaza, je leta 1912 dokazal Litllewood. Njeno bistvo je, da Riemannovo hipotezo lahko prevedemo na ekvivalentno trditev, ki opisuje rast Mertensove funkcije. 

\begin{trditev}
Za vsak $\epsilon>0$ velja \( M(n) = o(n^{1/2+\epsilon}) \) natanko tedaj, ko velja Riemannova hipoteza.
\end{trditev}

Pred drugo ekvivalenco Riemannove hipoteze potrebujemo definicijo, ki sledi.

\begin{definicija}
Naj bosta $L(n)$ dolžina Fareyevega zaporedja $F_{n}$ in $r_{v}$ njegov v-ti element. Definiramo razliko
\[ \delta_{v}= r_{v}-v/L(n). \]
\end{definicija}

\begin{primer}
Vzemimo Fareyevo zaporedje reda $4$ in poglejmo nekaj razlik. 

\(F_4 = \left \{\frac{0}{1}, \frac{1}{4}, \frac{1}{3}, \frac{1}{2}, \frac{2}{3}, \frac{3}{4}, \frac{1}{1} \right \}, \)

\( L(4) = 7, \)

\( \delta_{1}= \frac{0}{1} - \frac{1}{7} = -\frac{1}{7}, \)

\( \delta_{2}= \frac{1}{4} - \frac{2}{7} = -\frac{1}{28}, \)

\( \delta_{7}= \frac{1}{1} - \frac{7}{7} = 0. \)
\end{primer}

\begin{trditev}[Franel -- Landau, 1924]
Za vsak $\epsilon>0$ velja \( \sum_{v=1}^{L(n)}|\delta_{v}| = o(n^{1/2+\varepsilon}) \) natanko tedaj, ko velja Riemannova hipoteza.
\end{trditev}

Dokaz zgornje trditve izpustimo. Pač pa bomo v nadaljevanju dokazali povezavo med navedenima ekvivalenčnima formulacijama Riemannove hipoteze, ki jo lahko združimo v izrek.

\begin{izrek}
Naj bo $\varepsilon > 0$. \( \sum_{v=1}^{L(n)}|\delta_{v}| = o(n^{1/2+\varepsilon}) \) velja tedaj in le tedaj, ko velja \( M(n) = o(n^{1/2+\varepsilon}). \)
\end{izrek}

Naslednja lema bo ključ do dokaza zgornjega izreka.

\begin{lema}
\label{Lema:RiemFar}
Realna funkcija $f$ naj bo definirana na intervalu $[0,1]$. Naj bodo $r_{v}$ elementi Fareyevega zaporedja reda $n$, $r_{0}=0$ in $r_{L(n)}=1$\footnote{Zgolj zaradi preglednejšega zapisa bomo za potrebe dokaza Fareyevo zaporedje opazovali od neničelnega člena dalje. Tako bo element $r_{v}$ zdaj predstavljal element $r_{v+1}$ v običajni notaciji, vrednost $L(n)$ pa se bo zmanjšala za $1$.}. Tedaj velja enakost
\begin{equation}
	\sum_{v=1}^{L(n)} f(r_v) = \sum_{k=1}^{\infty} \sum_{j=1}^{k} f \left(\frac{j}{k} \right) M \left(\frac{n}{k} \right).
\end{equation}
\end{lema}

% Dokaz implikacije v desno
\subsubsection{Dokaz implikacije v desno}
%
Naj bo $\varepsilon > 0$. Naj bo $u \in [0,1]$ in definirajmo funkcijo $f(u) = e^{2\pi iu}$. Uporabimo lema~\ref{Lema:RiemFar} in funkcijo vstavimo v enakost~(14); dobimo
\begin{equation}
	\sum_{v=1}^{L(n)} e^{2\pi i r_{v}} = \sum_{k=1}^{\infty} \sum_{j=1}^{k} e^{2\pi i \frac{j}{k}} M \left(\frac{n}{k} \right).
\end{equation}
Vemo, da je $\sum_{j=1}^{k} e^{2\pi i \frac{j}{k}} = 0$ za $k \geq 2$, za $k=1$ pa se vsota poenostavi v $e^{2\pi i} = 1$.
Enakost~(15) zato prepišemo v 
\[ M(n) = \sum_{v=1}^{L(n)} e^{2\pi i r_{v}} = \sum_{v=1}^{L(n)} e^{2\pi i \left(\frac{v}{L(n)} + \delta_{v} \right)} 
	   = \sum_{v=1}^{L(n)} e^{ \frac{2\pi i v}{L(n)} } \left(e^{2\pi i \delta_{v}} - 1 \right) + \sum_{v=1}^{L(n)} e^{ \frac{2\pi iv}{L(n)} }. \]
Ker je $L(n) > 1$, je zadnji sumand enak $0$. Sedaj ocenimo absolutno vrednost zgornjega izraza:
\begin{align*} 
|M(n)| &\leq \sum_{v=1}^{L(n)} \left|e^{ \frac{2\pi i v}{L(n)} } \right| \left|e^{2\pi i \delta_{v}} - 1 \right| 
	= \sum_{v=1}^{L(n)} \left|e^{2\pi i \delta_{v}} - 1 \right|
	= \sum_{v=1}^{L(n)} \left|e^{\pi i \delta_{v} } \right| \left|e^{\pi i \delta_{v}} - e^{-\pi i \delta_{v}}\right| \\
	&= 2 \sum_{v=1}^{L(n)} |\sin(\pi \delta_{v})| \leq 2 \sum_{v=1}^{L(n)} |\delta_{v}| \pi = 2 \pi \sum_{v=1}^{L(n)} |\delta_{v}| \\
	&\leq 2 \pi K(\varepsilon) n^{1/2 + \varepsilon} = K'(\varepsilon) n^{1/2 + \varepsilon}.
\end{align*}
V zadnji neenakosti smo uporabili predpostavko $ \sum_{v=1}^{L(n)}|\delta_{v}| = o(n^{1/2+\varepsilon}) $, kar je ekvivalentno 
$ \sum_{v=1}^{L(n)}|\delta_{v}| \leq K(\varepsilon) n^{1/2+\varepsilon}$ za neko konstano $K$, ki je odvisna od $\varepsilon$.
Od tod sledi, da je $M(n) = o(n^{1/2 + \varepsilon})$.

% Pomožne definicije
\subsubsection{Pomožne definicije}
Preden se lotimo dokazovanja obratne implikacije, potrebujemo nekaj novih pojmov. Navedli bomo le najpomembnejše rezultate, ki jih bomo v dokazu potrebovali. Izpeljave niso pretežke, vendar jih bomo tokrat izpustili, saj bi precej povečale obseg dela. Povzeti so po \cite[poglavje 6.2]{zetafunction}.

\begin{definicija}
Naj bo $n \in \mathbb{N} \cup \{0\}.$ \emph{$n$-ti Bernoullijev polinom} $B_{n}$ je polinom stopnje $n$, ki ustreza zvezi
\begin{equation}
\int_{x}^{x+1} B_{n}(t) dt = x^{n}.
\end{equation}
\end{definicija}

\begin{primer}
Zapišimo nekaj Bernoullijevih polinomov najnižjih stopenj.

\( B_{0}(x) = 1, \)

\( B_{1}(x) = x - \frac{1}{2}, \)

\( B_{2}(x) = x^2 - x - \frac{1}{6}, \)

\( B_{3}(x) = x^3 - \frac{3}{2} x^2 + \frac{1}{2} x. \)
\end{primer}

\begin{definicija}
Naj bo $B_{n}$ $n$-ti Bernoullijev polinom. Pripadajoča funkcija $ \bar{B_{n}} $ je definirana kot $ \bar{B_{n}}(x) = B_{n}(x - \lfloor x \rfloor). $
\end{definicija}

\begin{opomba}
Iz definicije sledi, da je $ \bar{B_{n}} $ periodična funkcija.
\end{opomba}

\begin{primer}
Funkcijo $ \bar{B_{1}}(x) = x - \lfloor x \rfloor - \frac{1}{2} $ bomo potrebovali v dokazu.
\end{primer}

\begin{trditev}
Naj bo $k \in \mathbb{N}$. Tedaj je
\begin{equation}
\label{Eq:B_{n}(ku)}
B_{n}(ku) = k^{n-1} \left ( B_{n}(u) + B_{n} \left( u + \frac{1}{k} \right) + \dots + B_{n} \left( u + \frac{k-1}{k} \right) \right).
\end{equation}
\end{trditev}

\begin{dokaz}
Dokaz za $k=2$ se nahaja v \cite[poglavje 6.2, str.~102 -- 103]{zetafunction}. Za splošen $k$ je dokaz analogen.
\end{dokaz}

% Dokaz implikacije v levo
\subsubsection{Dokaz implikacije v levo}
Spomnimo se funkcije $ \bar{B_{1}}(x) = x - \lfloor x \rfloor - \frac{1}{2} $ in enačbe~\ref{Eq:B_{n}(ku)}. Izberimo $n=1$. Funkcija $\bar{B_{1}}$ je periodična s periodo $1$, zato je za $x=ku$ dovolj obravnavati vrednosti $0 \leq u \leq \frac{1}{k}$. Za te vrednosti pa se $\bar{B_{1}}$ ujema s funkcijo $B_{1}$. Enačba~\ref{Eq:B_{n}(ku)} zato dobi obliko
\begin{align}
\label{Eq:bar{B_{1}}(ku)}
\bar{B_{1}}(ku) &= \bar{B_{1}}(u) + \bar{B_{1}} \left( u + \frac{1}{k} \right) + \dots + \bar{B_{1}} \left( u + \frac{k-1}{k} \right) \nonumber \\
			&= \bar{B_{1}} \left( u + \frac{1}{k} \right) + \bar{B_{1}} \left( u + \frac{2}{k} \right) + \dots + \bar{B_{1}} \left( u + 1 \right).
\end{align}

Ključni korak dokaza je, da v enakost iz leme~\ref{Lema:RiemFar} vstavimo funkcijo $ \bar{B_{1}}$, pri čemer uporabimo zvezo~\ref{Eq:bar{B_{1}}(ku)}. Označimo
\begin{align}
G(u) &= \sum_{v=1}^{L(n)} \bar{B_{1}}(u+r_v) = \sum_{k=1}^{\infty} \sum_{j=1}^{k} \bar{B_{1}} \left(u + \frac{j}{k} \right) M \left(\frac{n}{k} \right)
	= \sum_{k=1}^{\infty} \bar{B_{1}}(ku) M \left(\frac{n}{k} \right)
\end{align}
in dobimo dva izraza za funkcijo $G$. Izračunali bomo integral 
\begin{equation}
I = \int_{0}^1 G(u)^2 du.
\end{equation}

% 1. primer
\underline{1.~primer:} $G(u) = \sum_{v=1}^{L(n)} \bar{B_{1}}(u+r_v)$.
Oglejmo si, kakšna je funkcija $G$. Razpišimo zgornjo zvezo:
\begin{align}
G(u) &= \sum_{v=1}^{L(n)} \bar{B_{1}}(u+r_v) = \sum_{v=1}^{L(n)} \left(u+r_v - \lfloor u+r_v \rfloor - \frac{1}{2} \right) \nonumber \\
       &= L(n)u + \sum_{v=1}^{L(n)}r_v - \sum_{v=1}^{L(n)} \lfloor u+r_v \rfloor - \frac{L(n)}{2},
\end{align}
in opazujmo člen s spodnjim celim delom. Ker so členi Fareyevega zaporedja razen $r_{L(n)}=1$ simetrično razporejeni okrog vrednosti $\frac{1}{2}$, lahko zapišemo $\sum_{v=1}^{L(n)} \lfloor u+r_v \rfloor = \sum_{v=1}^{L(n)} \lfloor u+1-r_v \rfloor$. Za $u\in[0,1]$ zavzame $\lfloor u+1-r_v \rfloor$ le celi števili $0$ in $1$, slednje je le v primeru, ko je $u=r_{v}$. Zato je 
\[
\sum_{v=1}^{L(n)} \lfloor u+1-r_v \rfloor = \left\{
\begin{array}{rl}
	1 & \textrm{če}\ u = r_{v}\\
	0 & \textrm{sicer}
\end{array},
\right.
\]
kar pomeni, da ima funkcija $G$, evaluirana v elementih Fareyevega zaporedja, skok za $-1$. Med Fareyevima sosedoma, torej na intervalu $[r_{v-1},r_{v}]$, je $G$ linearna funkcija spremenljivke $u$ s koeficientom $L(n)$.
Zaradi simetrije členov Fareyevega zaporedja velja še $\sum_{v=1}^{L(n)-1} \bar{B_{1}}(r_v) = 0$. Od tod izračunamo desno limito funkcije $G$ v točki $0$,
\[ \lim_{u \to 0} G(u) = \lim_{u \to 0} \bar{B_{1}}(u+1) = -\frac{1}{2}. \]
Funkcija $G$ se na intervalu $[r_{v-1},r_{v}]$ izraža s predpisom
\[ G(u) = L(n)u - v - \frac{1}{2}. \]
%
Izrazimo še 
\[ L(n)r_{v} = L(n) \left( r_{v}-\frac{v}{L(n)}+\frac{v}{L(n)} \right) = L(n) \delta_{v} + v, \]
\[ L(n)r_{v}-v+\frac{1}{2} = L(n) \delta_{v} + \frac{1}{2} \textrm{ , } L(n)r_{v-1}-v+\frac{1}{2} = L(n) \delta_{v-1} - \frac{1}{2}. \]
%
Sedaj lahko izračunamo integral 
\begin{align}
\label{Eq:I1}
I &\stackrel{(1)}{=} \sum_{v=1}^{L(n)} \int_{r_{v-1}}^{r_{v}} \left( L(n)u-v-\frac{1}{2}+1 \right)^2 du \nonumber \\
  &= \sum_{v=1}^{L(n)} \frac{(L(n)u-v+\frac{1}{2})^3}{3L(n)} \bigg \arrowvert_{r_{v-1}}^{r_{v}} \nonumber \\
  &= \frac{1}{3L(n)} \sum_{v=1}^{L(n)} \left( \left(L(n) \delta_{v} + \frac{1}{2} \right)^3 - \left(L(n) \delta_{v-1} - \frac{1}{2} \right)^3 \right) \nonumber \\
  &\stackrel{(2)}{=} \frac{1}{3L(n)} \sum_{v=1}^{L(n)} \left( \left(L(n) \delta_{v} + \frac{1}{2} \right)^3 - \left(L(n) \delta_{v} - \frac{1}{2} \right)^3 \right) \nonumber \\
  &= \frac{1}{3L(n)} \sum_{v=1}^{L(n)} \left(3 \left(L(n) \delta_{v} \right)^2 + \frac{1}{4} \right) \nonumber \\
  &= L(n) \sum_{v=1}^{L(n)} \delta_{v}^2 + \frac{1}{12L(n)} L(n) \nonumber \\
  &= L(n) \sum_{v=1}^{L(n)} \delta_{v}^2 + \frac{1}{12}.
\end{align}
Zaradi privzetka, da je $r_{0}=0$ in $r_{L(n)}=1$, enakost~(1) res drži.
V enakosti~(2) smo upoštevali, da je $\delta_{L(n)}=0$ in $L(n) \delta_{0}-\frac{1}{2} = L(n) \delta_{L(n)}-\frac{1}{2}.$

% 2. primer
\underline{2.~primer:} $G(u) = \sum_{k=1}^{\infty} \bar{B_{1}}(ku) M \left(\frac{n}{k} \right)$.
Računajmo integral
\begin{align}
\label{Eq:I_{ab}}
I &= \int_{0}^1 \sum_{k=1}^{\infty} \left( \bar{B_{1}}(ku) M \left(\frac{n}{k} \right) \right)^2 du \nonumber \\
  &= \int_{0}^1 \sum_{a=1}^{\infty} \sum_{b=1}^{\infty} \bar{B_{1}}(au) \bar{B_{1}}(bu) M \left(\frac{n}{a} \right) M \left(\frac{n}{b} \right) du \nonumber \\
  &= \sum_{a=1}^{\infty} \sum_{b=1}^{\infty} M \left(\frac{n}{a} \right) M \left(\frac{n}{b} \right) \int_{0}^1 \bar{B_{1}}(au) \bar{B_{1}}(bu) du,
\end{align}
% 
in označimo $I_{ab} = \int_{0}^1 \bar{B_{1}}(au) \bar{B_{1}}(bu) du.$ Vrednost integrala bomo izračunali v treh korakih.

Naj bo $b=1$. Tedaj je
\begin{align}
I_{a1} &= \int_{0}^1 \bar{B_{1}}(au) \bar{B_{1}}(u) du
	\stackrel{(1)}{=} \frac{1}{a} \int_{0}^a \bar{B_{1}}(v) \bar{B_{1}} \left(\frac{v}{a} \right) dv \nonumber \\
	&\stackrel{(2)}{=} \frac{1}{a} \sum_{k=0}^{a-1} \int_{0}^1 \bar{B_{1}}(k+t) \bar{B_{1}} \left(\frac{k}{a}+\frac{t}{a} \right) dt \nonumber \\
	&\stackrel{(3)}{=} \frac{1}{a} \int_{0}^1 \bar{B_{1}}(t) \bar{B_{1}} \left(a\frac{t}{a} \right) dt
	\stackrel{(4)}{=} \frac{1}{a} \int_{0}^1 \left( t-\frac{1}{2} \right)^2 dt \nonumber \\
	&= \frac{1}{3a} \left( t-\frac{1}{2} \right)^3 \bigg \arrowvert_{0}^{1}
	= \frac{1}{12a}.
\end{align}
V enakostih~(1) in (2) smo zaporedoma uvedli novi spremenljivki $v=au$ in $v=k+t$. V enakosti~(3) smo upoštevali periodičnost funkcije $\bar{B_{1}}$ in zvezo \ref{Eq:bar{B_{1}}(ku)}. V enakosti~(4) smo upoštevali, da na intervalu $[0,1]$ velja $\bar{B_{1}}(t) = t-\frac{1}{2}$.

Oglejmo si primer, ko sta $a$ in $b$ tuji si števili.
\begin{align}
I_{ab} &= \int_{0}^1 \bar{B_{1}}(au) \bar{B_{1}}(bu) du
	= \frac{1}{a} \int_{0}^a \bar{B_{1}}(v) \bar{B_{1}} \left(\frac{bv}{a} \right) dv \nonumber \\
	&= \frac{1}{a} \sum_{k=0}^{a-1} \int_{0}^1 \bar{B_{1}}(k+t) \bar{B_{1}} \left(\frac{bk}{a}+\frac{bt}{a} \right) dt \nonumber \\
	&\stackrel{(5)}{=} \frac{1}{a} \int_{0}^1 \bar{B_{1}}(t) \bar{B_{1}} \left(a\frac{bt}{a} \right) dt
	= \frac{1}{a} I_{1b} 
	= \frac{1}{12ab} 
\end{align}
V enakosti~(5) smo uporabili dejstvo, da vrednosti $\frac{bk}{a}$ za $k \in \{0,1, \dots, a-1 \}$ pretečejo vse vrednosti iz množice $\{0, \frac{1}{a}, \dots, \frac{a-1}{a}\}$.

Naj bo sedaj $c$ največji skupni delitelj $a$ in $b$. Zapišimo $a = c \alpha$ in $b = c \beta$ (seveda sta $\alpha$ in $\beta$ tuji si števili).
\begin{align}
I_{ab} &= \int_{0}^1 \bar{B_{1}}(c \alpha u) \bar{B_{1}}(c \beta u) du
	\stackrel{(6)}{=} \frac{1}{c} \int_{0}^c \bar{B_{1}}(\alpha t) \bar{B_{1}}(\beta t) dt \nonumber \\
	&\stackrel{(7)}{=} I_{\alpha \beta}
	= \frac{1}{12 \alpha \beta}
	= \frac{c^2}{12ab}
\end{align}
V enakosti~(6) smo uvedli novo spremenljivko $t=cu$.
Enakost~(7) sledi iz računa
\begin{align*}
\int_{k}^{k+1} \bar{B_{1}}(\alpha t) \bar{B_{1}}(\beta t) dt &= \int_{0}^{1} \bar{B_{1}}(\alpha s + \alpha k) \bar{B_{1}}(\beta s + \beta k) ds \\
 	&= \int_{0}^{1} \bar{B_{1}}(\alpha s) \bar{B_{1}}(\beta s) ds = I_{\alpha \beta}.
\end{align*}
%
Izračunano vrednost integrala $I_{ab}$ vstavimo v enakost \ref{Eq:I_{ab}} in dobimo
\begin{equation}
\label{Eq:I2}
I = \sum_{a=1}^{\infty} \sum_{b=1}^{\infty} M \left(\frac{n}{a} \right) M \left(\frac{n}{b} \right) \frac{c^2}{12ab}.
\end{equation}

% Ocena in uporaba predpostavke
Naj bo $\epsilon>0$. Predpostavljamo, da je $M(n) = o(x^{1/2+\epsilon})$, kar pomeni, da obstaja taka konstanta $C$, odvisna od $\epsilon$, da za poljuben $n$ velja $M(n) < C(\epsilon) n^{1/2+\epsilon}$. Če to uporabimo v enačbi~\ref{Eq:I2}, dobimo
\begin{align}
I &< \sum_{a=1}^{\infty} \sum_{b=1}^{\infty} C(\epsilon)^2 \left( \frac{n}{a} \right)^{1/2+\epsilon} \left( \frac{n}{b} \right)^{1/2+\epsilon} \frac{c^2}{12ab} \nonumber \\
  &= n^{1+2\epsilon} \frac{C(\epsilon)^2}{12} \sum_{a=1}^{\infty} \sum_{b=1}^{\infty} \frac{c^2}{a^{3/2+\epsilon} b^{3/2+\epsilon}} \nonumber \\
  &= n^{1+2\epsilon} \frac{C(\epsilon)^2}{12} \sum_{a=1}^{\infty} \sum_{b=1}^{\infty} \frac{c^2}{\alpha^{3/2+\epsilon} \beta^{3/2+\epsilon} c^{3+2\epsilon}} \nonumber \\
  &\stackrel{(8)}{<} n^{1+2\epsilon} K_{1}(\epsilon) \sum_{\alpha=1}^{\infty} \sum_{\beta=1}^{\infty} \sum_{c=1}^{\infty} \frac{1}{\alpha^{3/2} \beta^{3/2} c^{1+2\epsilon}} \nonumber \\
  &= K_{2}(\epsilon) n^{1+2\epsilon}.
\end{align}
V neenakosti~(8) smo vsoto po tujih si številih $\alpha$ in $\beta$ zamenjali z vsoto po vseh vrednostih $\alpha$ in $\beta$.

Enakost~\ref{Eq:I1} implicira
\[ I = L(n) \sum_{v=1}^{L(n)} \delta_{v}^2 + \frac{1}{12} < K_{2}(\epsilon) n^{1+2\epsilon}, \]
od koder sledi
\[ \sum_{v=1}^{L(n)} \delta_{v}^2 < K_{3}(\epsilon) n^{1+2\epsilon}. \]
Cauchy-Schwarzova neenakost nam da končno oceno
\begin{align*}
\sum_{v=1}^{L(n)} |\delta_{v}| &= \left| \sum_{v=1}^{L(n)} (\pm1)\delta_{v} \right|
	\leq \sqrt{\sum_{v=1}^{L(n)} (\pm1)^2}  \sqrt{\sum_{v=1}^{L(n)} \delta_{v}^2} 
	= L(n)^{1/2} \sqrt{\sum_{v=1}^{L(n)} \delta_{v}^2}
	< K(\epsilon) n^{1+2\epsilon},
\end{align*}
kar smo želeli pokazati.

%
%%%%%%%%%%%%%%%%%%%%%%%%%%%%%%%%%%%%%%%%%%%%%%%%%%%%%%%%%%%%%%%%%%%%%%%%%%%%%%%%%%%%%%%%%%%%%%%%%%%%%%%%%%%%%%%%%%%%%%%%%%%%%%%%%%%%%%%%%%%%%%%%%%


\section*{Slovar strokovnih izrazov}

\geslo{}{}
\geslo{}{}

% seznam uporabljene literature
\begin{thebibliography}{99}

\bibitem{fareyproject} J.~Ainsworth, M.~Dawson, J.~Pianta in J.~Warwick, \emph{The Farey sequence}, diplomsko delo, School of Mathematics, University of Edinburgh, 2012; dostopno tudi na \url{https://www.maths.ed.ac.uk/~v1ranick/fareyproject.pdf}.

\bibitem{zetafunction} H.~M.~Edwards, \emph{Riemann's zeta function}, Academic Press, Inc., New York, 1974.

\bibitem{ford} L.~R.~Ford, \emph{Fractions}, v: The American Mathematical Monthly (ur.\ E.~J.~Moulton) \textbf{45}, Mathematical Association of America, 1938, str.\ 586--601.

\bibitem{motifofmath} S.~B.~Guthery, \emph{A motif of mathematics}, Docent Press, Boston, 2011; dostopno tudi na \url{https://www.maths.ed.ac.uk/~v1ranick/papers/farey.pdf}.

\bibitem{hardy} G.~H.~Hardy in E.~M.~Wright, \emph{An introduction to the theory of numbers}, 4th ed., Oxford University Press, Oxford, 1960.

\bibitem{TNBook} A.~Hatcher, \emph{Topology of numbers}, verzija junij 2018, [ogled 31.~10.~2018], dostopno na \url{https://pi.math.cornell.edu/~hatcher/TN/TNpage.html}.

\end{thebibliography}

\end{document}

